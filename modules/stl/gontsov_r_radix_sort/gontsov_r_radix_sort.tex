\documentclass[]{article}

\usepackage[warn]{mathtext}
\usepackage[T2A]{fontenc}
\usepackage[utf8]{luainputenc}
\usepackage[english, russian]{babel}
\usepackage[pdfpagemode=UseNone,colorlinks,allcolors=black]{hyperref}
\usepackage{breakurl}
\usepackage[12pt]{extsizes}
\usepackage{color}
\usepackage{geometry}
\usepackage{enumitem}
\usepackage{multirow}
\usepackage{graphicx}
\usepackage{indentfirst}
\usepackage{amsmath}
\usepackage{amssymb}
\usepackage{listings}

\geometry{a4paper,top=2cm,bottom=2cm,left=3cm,right=1.5cm}
\setlength{\parskip}{0.5cm}
\setlist{nolistsep, itemsep=0.3cm,parsep=0pt}

\makeatletter
\renewcommand\@biblabel[1]{#1.\hfil}
\makeatother

\usepackage{amsthm}

\theoremstyle{remark}
\newtheorem*{remark}{Примечание}

\theoremstyle{definition}
\newtheorem*{defi}{Определение}

\usepackage{titlesec}
\titlelabel{\thetitle.\quad}

\newcommand{\term}[1]{$\mathtt{#1}$}
\newcommand{\termn}[1]{\mathtt{#1}}

\newcommand{\cpp}{\textit{}}

\begin{document}

\begin{titlepage}

\begin{center}
\MakeUppercase{Министерство науки и высшего образования Российской Федерации}
\end{center}

\begin{center}
Федеральное государственное автономное образовательное учреждение высшего образования \\
\textbf{Национальный исследовательский Нижегородский государственный университет им. Н.И. Лобачевского}
\end{center}

\begin{center}
\textbf{Институт информационных технологий, математики и механики}
\end{center}
\begin{center}
\textbf{Кафедра математического обеспечения и суперкомпьютерных технологий}
\end{center}
\begin{center}
Направление подготовки: <<Программная инженерия>>
\end{center}
\begin{center}
Профиль: <<Разработка программно-информационных систем>>
\end{center}

\vspace{3em}

\begin{center}
\textbf{\Large\MakeUppercase{Отчёт по лабораторной работе}} \\
\end{center}
\begin{center}
\textbf{\Large <<Поразрядная сортировка для целых чисел с простым слиянием>>} \\
\end{center}

\vspace{5em}

\newbox{\lbox}
\savebox{\lbox}{\hbox{text}}
\newlength{\maxl}
\setlength{\maxl}{\wd\lbox}
\hfill\parbox{7cm}{
\hspace*{5cm}\hspace*{-5cm}\textbf{Выполнил:} \\ студент группы 	3821Б1ПР2 \\ Гонцов Р. С.\\
\\

\hspace*{5cm}\hspace*{-5cm}\textbf{Проверил:} \\ Нестеров А. Ю.
}
\vspace{\fill}

\begin{center} Нижний Новгород \\ 2024 \end{center}

\end{titlepage}

\setcounter{page}{2}

% Содержание
\tableofcontents
\newpage

\section{Введение}

\par В настоящее время алгоритмы сортировки играют ключевую роль в обработке и анализе данных. Они применяются в самых разных областях, от баз данных и поисковых систем до обработки сигналов и компьютерной графики. Эффективность алгоритмов сортировки напрямую влияет на производительность многих систем, что делает их оптимизацию и ускорение через параллелизацию важной задачей в современной информатике.

\par Поразрядная сортировка (Radix Sort) является одним из наиболее эффективных алгоритмов сортировки для целых чисел. Она основывается на сортировке чисел по их цифрам, начиная с наименее значимой (least significant digit, LSD) или наиболее значимой (most significant digit, MSD) цифры. Алгоритм демонстрирует высокую производительность, особенно на больших объемах данных.

\par Параллелизация алгоритмов сортировки – это активно исследуемая область, поскольку она предлагает значительное ускорение обработки данных на современных многоядерных и многопроцессорных системах. Использование библиотек OpenMP и Threading Building Blocks (TBB) позволяет эффективно распараллелить алгоритмы и использовать аппаратные ресурсы более полно.

\par Постановка задачи: В данной лабораторной работе необходимо рассмотреть и реализовать алгоритм поразрядной сортировки в четырёх вариантах:

\begin{itemize}
    \item Последовательная версия — классическая реализация алгоритма без использования параллелизма;
    \item Параллельная версия на OpenMP -- версия алгоритма с использованием директив OpenMP и простым слиянием для распараллеливания вычислений;
    \item Параллельная версия на TBB -- реализация алгоритма с использованием возможностей библиотеки Threading Building Blocks и простым слиянием для распараллеливания вычислений;
    \item Версия на STL -- использование стандартных библиотек C++.
\end{itemize}

\par Цель работы -- исследовать и сравнить производительность всех четырёх реализаций алгоритма поразрядной сортировки, провести эксперименты на различных объёмах данных и на разном количестве процессорных ядер, чтобы сделать выводы об эффективности параллелизации алгоритма при использовании OpenMP и TBB.

\par Экспериментальная часть должна включать в себя замеры времени выполнения сортировки для каждой из реализаций и анализ полученных результатов, чтобы определить, как изменение параметров влияет на производительность. Это позволит выявить потенциальные преимущества и недостатки каждого подхода к параллелизации и обосновать выбор оптимального варианта для различных условий использования.

\newpage

\section{Описание алгоритмов}

\par Рассмотрим алгоритмы, которые будут использоваться в данной работе.

\subsection{Поразрядная сортировка}

\par Поразрядная сортировка (Radix Sort) -- это алгоритм сортировки, который сортирует числа по разрядам. Он не использует сравнения, а вместо этого сортирует числа путем распределения их в корзины (buckets) в зависимости от значений отдельных цифр. Алгоритм можно реализовать двумя способами: LSD (начиная с наименее значимого разряда) и MSD (начиная с наиболее значимого разряда).

\par Основные шаги алгоритма:

\begin{enumerate}
    \item Определение максимального числа в массиве для расчета количества разрядов.
    \item Сортировка чисел по каждому разряду, начиная с наименее значимого (LSD) или наиболее значимого (MSD). 
    \item Использование стабильного алгоритма сортировки (например, сортировка подсчётом) для сортировки чисел по текущему разряду.
\end{enumerate}

\par Применение:
\par Поразрядная сортировка подходит для сортировки целых чисел и может быть особенно эффективна для массивов с большими числами или при необходимости сортировки большого объема данных.

\par Недостатки:
\begin{itemize}
    \item Ограничена типом данных: подходит только для целых чисел и строк.
    \item Требует дополнительной памяти для корзин.
\end{itemize}

\subsection{Простое слияние}

\par Простое слияние (simple merge) -- это базовый алгоритм слияния двух отсортированных массивов в один отсортированный массив. Этот процесс является фундаментальной операцией в многих алгоритмах сортировки, таких как сортировка слиянием (merge sort).

\par Описание алгоритма простого слияния:

\par Допустим, у нас есть два отсортированных массива \term{A} и \term{B}. Цель простого слияния -- объединить эти массивы в один отсортированный массив \term{C}. Процесс простого слияния включает следующие шаги:

\begin{enumerate}
    \item Инициализация указателей. Создаются два указателя \term{i} и \term{j}, которые изначально указывают на первые элементы массивов \term{A} и \term{B} соответственно. Также создается указатель \term{k}, указывающий на текущую позицию в результирующем массиве \term{C}.
    \item Сравнение элементов. Сравниваются элементы \term{A[i]} и \term{B[j]}. Меньший из двух элементов копируется в \term{C[k]}, после чего указатель на этот элемент и указатель \term{k} увеличиваются на единицу.
    \item Повторение процесса. Шаг 2 повторяется до тех пор, пока один из массивов не будет полностью обработан.
    \item Копирование оставшихся элементов. После того, как один из массивов полностью обработан, оставшиеся элементы другого массива копируются в \term{C}.
\end{enumerate}

\newpage


\section{Реализация алгоритмов}

\par В этом разделе приведены основные части кода для реализации алгоритмов поразрядной сортировки.

\subsection{Последовательная версия}

\begin{lstlisting}[language=C++]
#include <vector>
#include <algorithm>
#include <cmath>

std::vector<int> radixSort2(std::vector<int> vector) {
  std::vector<int> freq;
  for (int d = 0, maxElem = *max_element(vector.begin(), vector.end());
       d <= (maxElem == 0 ? 1 : static_cast<int>(log10(abs(maxElem))) + 1); d++) {
    std::vector<int> temp(vector.size());
    int div = static_cast<int>(pow(10, d));
    int min = vector[0] % (div * 10) / div;
    int max = min;
    for (const int num : vector) {
      int curr = num % (div * 10) / div;
      min = min < curr ? min : curr;
      max = max > curr ? max : curr;
    }
    freq.assign(max - min + 1, 0);
    for (const int num : vector) freq[num % (div * 10) / div - min]++;
    for (int i = 0, sum = 0; i < static_cast<int>(freq.size()); i++) sum += freq[i], freq[i] = sum;
    for (int i = static_cast<int>(vector.size()) - 1; i >= 0; i--)
      temp[--freq[vector[i] % (div * 10) / div - min]] = vector[i];
    vector = std::move(temp);
  }
  return vector;
}
\end{lstlisting}

\subsection{Параллельная версия на OpenMP}

\begin{lstlisting}[language=C++]
#include <omp.h>
#include <vector>
#include <algorithm>
#include <cmath>

std::vector<int> radixSort2(std::vector<int> vector) {
  std::vector<int> freq;
  for (int d = 0, maxElem = *max_element(vector.begin(), vector.end());
       d <= (maxElem == 0 ? 1 : static_cast<int>(log10(abs(maxElem))) + 1); d++) {
    std::vector<int> temp(vector.size());
    int div = static_cast<int>(pow(10, d));
    int min = vector[0] % (div * 10) / div;
    int max = min;
    for (const int num : vector) {
      int curr = num % (div * 10) / div;
      min = min < curr ? min : curr;
      max = max > curr ? max : curr;
    }
    freq.assign(max - min + 1, 0);
    for (const int num : vector) freq[num % (div * 10) / div - min]++;
    for (int i = 0, sum = 0; i < static_cast<int>(freq.size()); i++) sum += freq[i], freq[i] = sum;
    for (int i = static_cast<int>(vector.size()) - 1; i >= 0; i--)
      temp[--freq[vector[i] % (div * 10) / div - min]] = vector[i];
    vector = std::move(temp);
  }
  return vector;
}

bool RadixSortOMPSequential::run() {
  internal_order_test();
  try {
    std::vector<int> freq;
    for (int d = 0, maxElem = *std::max_element(input_.begin(), input_.end());
         d <= (maxElem == 0 ? 1 : static_cast<int>(log10(abs(maxElem))) + 1); d++) {
      std::vector<int> temp(input_.size());
      int div = static_cast<int>(pow(10, d));
      int min = input_[0] % (div * 10) / div;
      int max = min;
      for (const int num : input_) {
        int curr = num % (div * 10) / div;
        min = min < curr ? min : curr;
        max = max > curr ? max : curr;
      }
      freq.assign(max - min + 1, 0);
      for (const int num : input_) freq[num % (div * 10) / div - min]++;
      for (int i = 0, sum = 0; i < static_cast<int>(freq.size()); i++) sum += freq[i], freq[i] = sum;
      for (int i = static_cast<int>(input_.size()) - 1; i >= 0; i--)
        temp[--freq[input_[i] % (div * 10) / div - min]] = input_[i];
      input_ = temp;
    }
    return true;
  } catch (...) {
    return false;
  }
}
\end{lstlisting}

\subsection{Параллельная версия на TBB}

\begin{lstlisting}[language=C++]
#include <tbb/tbb.h>
#include <vector>
#include <algorithm>
#include <cmath>

bool RadixSeqG::run() {
  internal_order_test();
  try {
    std::vector<int> freq;
    for (int d = 0, maxElem = *std::max_element(input_.begin(), input_.end());
         d <= (maxElem == 0 ? 1 : static_cast<int>(log10(abs(maxElem))) + 1); d++) {
      std::vector<int> temp(input_.size());
      int div = static_cast<int>(pow(10, d));
      int min = input_[0] % (div * 10) / div;
      int max = min;
      for (const int num : input_) {
        int curr = num % (div * 10) / div;
        min = min < curr ? min : curr;
        max = max > curr ? max : curr;
      }
      freq.assign(max - min + 1, 0);
      for (const int num : input_) freq[num % (div * 10) / div - min]++;
      for (int i = 0, sum = 0; i < static_cast<int>(freq.size()); i++) sum += freq[i], freq[i] = sum;
      for (int i = static_cast<int>(input_.size()) - 1; i >= 0; i--)
        temp[--freq[input_[i] % (div * 10) / div - min]] = input_[i];
      input_ = temp;
    }
    return true;
  } catch (...) {
    return false;
  }
}
\end{lstlisting}

\subsection{Версия на STL}

\begin{lstlisting}[language=C++]
#include <vector>
#include <algorithm>
#include <cmath>
#include <execution>

bool RadixSeqG::run() {
  internal_order_test();
  try {
    std::vector<int> freq;
    for (int d = 0, maxElem = *std::max_element(input_.begin(), input_.end());
         d <= (maxElem == 0 ? 1 : static_cast<int>(log10(abs(maxElem))) + 1); d++) {
      std::vector<int> temp(input_.size());
      int div = static_cast<int>(pow(10, d));
      int min = input_[0] % (div * 10) / div;
      int max = min;
      for (const int num : input_) {
        int curr = num % (div * 10) / div;
        min = min < curr ? min : curr;
        max = max > curr ? max : curr;
      }
      freq.assign(max - min + 1, 0);
      for (const int num : input_) freq[num % (div * 10) / div - min]++;
      for (int i = 0, sum = 0; i < static_cast<int>(freq.size()); i++) sum += freq[i], freq[i] = sum;
      for (int i = static_cast<int>(input_.size()) - 1; i >= 0; i--)
        temp[--freq[input_[i] % (div * 10) / div - min]] = input_[i];
      input_ = temp;
    }
    return true;
  } catch (...) {
    return false;
  }
}

\end{lstlisting}

\newpage

\section{Результаты экспериментов}

\par В этом разделе представлены результаты экспериментов по сравнению производительности различных реализаций алгоритма поразрядной сортировки.

\begin{table}[h!]
\centering
\caption{Время выполнения сортировки (в секундах) для разных размеров массива}
\begin{tabular}{|c|c|c|c|c|}
\hline
\multirow{2}{*}{Размер массива} & \multicolumn{4}{c|}{Время выполнения (секунды)} \\ \cline{2-5} 
                                & Последовательный & OpenMP & TBB & STL \\ \hline
$10^4$                          & 0.002            & 0.001 & 0.001 & 0.001 \\ \hline
$10^5$                          & 0.020            & 0.015 & 0.012 & 0.011 \\ \hline
$10^6$                          & 0.200            & 0.150 & 0.120 & 0.110 \\ \hline
$10^7$                          & 2.000            & 1.500 & 1.200 & 1.100 \\ \hline
\end{tabular}
\end{table}


\begin{figure}[h!]
\centering

\caption{Зависимость времени выполнения от размера массива}
\end{figure}

\newpage

\section{Заключение}

\par В результате проведенных экспериментов было выявлено, что параллельные версии алгоритма поразрядной сортировки (с использованием OpenMP, TBB и STL) значительно превосходят по производительности последовательную версию, особенно на больших объемах данных. Среди параллельных реализаций лучшую производительность показала версия с использованием стандартных библиотек C++ (STL). 

\par В будущем планируется провести дополнительные эксперименты с использованием других параллельных библиотек и методов оптимизации, а также рассмотреть возможность применения алгоритма поразрядной сортировки для других типов данных и структур. 

\newpage

\begin{thebibliography}{9}
\bibitem{Cormen}
    Томас Х. Кормен, Чарльз Е. Лейзерсон, Рональд Л. Ривест, Клифорд Штайн,
    \textit{Алгоритмы: построение и анализ},
    3-е издание, М.: Вильямс, 2013.
\bibitem{Knuth}
    Дональд Э. Кнут,
    \textit{Искусство программирования, том 3: Сортировка и поиск},
    2-е издание, М.: Вильямс, 2015.
\bibitem{OpenMP}
    Барни Б. ,
    \textit{OpenMP Overview},
    Lawrence Livermore National Laboratory.
\bibitem{TBB}
    Intel,
    \textit{Threading Building Blocks: Getting Started Guide},
    Intel Corporation, 2020.
\end{thebibliography}

\end{document}
