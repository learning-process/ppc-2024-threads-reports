\documentclass{report}

\usepackage[warn]{mathtext}
\usepackage[T2A]{fontenc}
\usepackage[utf8]{luainputenc}
\usepackage[english, russian]{babel}
\usepackage[pdfpagemode=UseNone,colorlinks,allcolors=black]{hyperref}
\usepackage{tempora}
\usepackage[12pt]{extsizes}
\usepackage{listings}
\usepackage{color}
\usepackage{geometry}
\usepackage{enumitem}
\usepackage{multirow}
\usepackage{graphicx}
\usepackage{indentfirst}
\usepackage{amsmath}
\usepackage{soul}

\sethlcolor{gray}

\geometry{a4paper,top=2cm,bottom=2cm,left=2.5cm,right=1.5cm}
\setlength{\parskip}{0.5cm}
\setlist{nolistsep, itemsep=0.3cm,parsep=0pt}

\usepackage{listings}
\lstset{language=C++,
        basicstyle=\footnotesize,
		keywordstyle=\color{blue}\ttfamily,
		stringstyle=\color{red}\ttfamily,
		commentstyle=\color{green}\ttfamily,
		morecomment=[l][\color{red}]{\#}, 
		tabsize=4,
		breaklines=true,
  		breakatwhitespace=true,
  		title=\lstname,       
}

\makeatletter
\renewcommand\@biblabel[1]{#1.\hfil}
\makeatother

\begin{document}

\begin{titlepage}

\begin{center}
Министерство науки и высшего образования Российской Федерации
\end{center}

\begin{center}
Федеральное государственное автономное образовательное учреждение высшего образования \\
Национальный исследовательский Нижегородский государственный университет \newline им. Н.И. Лобачевского
\end{center}

\begin{center}
Институт информационных технологий, математики и механики \\
Кафедра математического обеспечения и суперкомпьютерных технологий \\
Направление подготовки: Программная инженерия
\end{center}

\begin{center}
\textbf{Отчет по курсу \\
\vspace{0.5em}
<<Параллельное программирование для систем с общей памятью>>} \\
\end{center}

\vspace{4em}

\begin{center}
\textbf{\LargeТема \\
\vspace{0.5em}
<<Линейная фильтрация изображений (вертикальное разбиение). Ядро Гаусса 3x3>>} \\
\end{center}

\vspace{4em}

\newbox{\lbox}
\savebox{\lbox}{\hbox{text}}
\newlength{\maxl}
\setlength{\maxl}{\wd\lbox}
\hfill\parbox{7cm}{
\hspace*{5cm}\hspace*{-5cm}\textbf{Выполнил:} \\ студент группы 3821Б1ПР2\\Кочетов Н. А.\\
\\
\hspace*{5cm}\hspace*{-5cm}\textbf{Проверил:}\\ аспирант\\Нестеров А. Ю.\\
}
\vspace{\fill}

\begin{center} Нижний Новгород \\ 2024 \end{center}

\end{titlepage}

\setcounter{page}{2}

% Содержание
\tableofcontents
\newpage

% Введение
\section*{Введение}
\addcontentsline{toc}{section}{Введение}
\par В эпоху стремительного развития технологий обработки цифровых изображений одной из ключевых задач является эффективное применение параллельных вычислений для повышения производительности алгоритмов. Линейная фильтрация, в частности с использованием ядра Гаусса размера 3x3, является распространенным методом улучшения качества изображений, устранения шумов и выделения важных деталей. Однако при обработке крупных изображений данный процесс может быть весьма ресурсоемким и требовать значительных вычислительных мощностей.

\par Целью настоящей работы является разработка и анализ эффективности параллельного алгоритма линейной фильтрации изображений с применением ядра Гаусса 3x3 и вертикального разбиения изображения на системах с общей памятью. Вертикальное разбиение позволяет распределить вычислительную нагрузку между несколькими потоками или процессами, что может существенно ускорить процесс фильтрации и повысить производительность.

\par В рамках данной работы будут рассмотрены теоретические основы линейной фильтрации изображений, описан параллельный алгоритм фильтрации с использованием ядра Гаусса и вертикального разбиения, а также проведены эксперименты по оценке эффективности и ускорения предложенного алгоритма на различных наборах данных и конфигурациях системы. Результаты исследования могут быть полезны для разработчиков программного обеспечения в области обработки изображений и компьютерного зрения, а также для исследователей, занимающихся параллельными вычислениями и оптимизацией алгоритмов.

\par В рамках задачи реализован алгоритм с использованием технологий: SEQ, TBB, STD.

\newpage

% Постановка задачи
\section*{Постановка задачи}
\addcontentsline{toc}{section}{Постановка задачи}
\par \textbf{В рамках данного задания необходимо выполнить следующие задачи:}
\begin{enumerate}
\item Реализовать три версии алгоритма линейной фильтрации изображений с использованием ядра Гаусса 3x3 и вертикального разбиения:
\begin{itemize}
\item Последовательная реализация (SEQ)
\item Параллельная реализация с использованием технологии TBB
\item Параллельная реализация с использованием стандартной библиотеки C++
\end{itemize}
\item Сравнить реализации и производительность версий.
\item Провести анализ полученных данных.
\end{enumerate}
\newpage

% Описание алгоритма
\section*{Описание алгоритма}
\addcontentsline{toc}{section}{Описание алгоритма}
\par Линейная фильтрация изображений с использованием ядра Гаусса 3x3 является широко применяемым методом для сглаживания изображений и удаления высокочастотных шумов. Данный алгоритм заключается в вычислении взвешенного среднего значения пикселей в окрестности каждого пикселя исходного изображения.

\par \textbf{Алгоритм линейной фильтрации с использованием ядра Гаусса 3x3 и вертикального разбиения:}

\begin{enumerate}
\item Разделить исходное изображение на вертикальные полосы.
\item Для каждой полосы выполнить следующие действия:
\begin{enumerate}
\item Применить ядро Гаусса 3x3 к каждому пикселю полосы, вычисляя взвешенное среднее значение соседних пикселей.
\item Обработать граничные пиксели, используя соответствующие значения из соседних полос.
\end{enumerate}
\item Объединить обработанные полосы в результирующее изображение.
\end{enumerate}

\par Вертикальное разбиение изображения позволяет распределить вычислительную нагрузку между несколькими потоками или процессами, что может значительно ускорить процесс фильтрации и повысить производительность.

\par В рамках данной работы будут реализованы три версии алгоритма линейной фильтрации с использованием ядра Гаусса 3x3 и вертикального разбиения:
\begin{itemize}
\item Последовательная реализация (SEQ)
\item Параллельная реализация с использованием технологии TBB
\item Параллельная реализация с использованием стандартной библиотеки C++
\end{itemize}
Будет проведено сравнение реализаций и анализ производительности каждой версии.

\newpage

% Описание схемы распараллеливания
\section* {Описание схемы распараллеливания}
\addcontentsline{toc}{section}{Описание схемы распараллеливания}
\par \textbf{Алгоритм линейной фильтрации изображений с использованием ядра Гаусса 3x3 и вертикального разбиения состоит из следующих основных этапов, которые были распараллелены:}
\begin{enumerate}
\item Копирование входных данных в изображение
\item Применение ядра Гаусса 3x3 к каждому пикселю изображения
\item Копирование обработанного изображения в выходные данные
\end{enumerate}
\vspace{2em}
\par \textbf{На втором этапе (применение ядра Гаусса 3x3) было распараллелено:}
\begin{enumerate}
\item Вычисление новых значений цветовых компонент каждого пикселя с использованием ядра Гаусса 3x3
\item Ограничение значений цветовых компонент до допустимого диапазона
\end{enumerate}
\vspace{1em}
\par Вертикальное разбиение изображения позволяет распределить вычислительную нагрузку между несколькими потоками или процессами, что может значительно ускорить процесс фильтрации и повысить производительность.

\newpage

\section* {Описание SEQ, OpenMP, TBB и STL версий}
\par В различных версиях алгоритма линейной фильтрации изображений с использованием ядра Гаусса 3x3 и горизонтального разбиения, основные участки кода, подлежащие распараллеливанию, остаются неизменными. Однако ключевое различие между версиями, такими как SEQ (последовательная версия), OpenMP, TBB (Threading Building Blocks) и STL (Standard Template Library), заключается в предоставляемых ими возможностях и методах распараллеливания.

\par \textbf{Версии реализации:}
\begin{enumerate}
\item В SEQ версии выполняется последовательное применение ядра Гаусса 3x3 к каждому пикселю изображения, без использования параллельных вычислений.

\item В TBB используется tbb::parallel\_for, что позволяет распараллелить вычисления новых значений цветовых компонент для каждого пикселя изображения, разбивая задачу на блоки и распределяя их между потоками.

\item В STL для параллельного выполнения используются объекты std::thread, которые создают потоки, обрабатывающие различные части изображения. Также используются методы std::thread::join для ожидания завершения выполнения всех потоков.
\end{enumerate}

\par Во всех версиях основными распараллеливаемыми участками кода являются:
\begin{enumerate}
\item Вычисление новых значений цветовых компонент для каждого пикселя изображения с использованием ядра Гаусса 3x3.
\item Ограничение значений цветовых компонент до допустимого диапазона.
\end{enumerate}

\par Различия между версиями заключаются в используемых средствах параллельного программирования и способах распределения вычислительной нагрузки между потоками.

\newpage

\section*{Результаты экспериментов}
\addcontentsline{toc}{section}{Результаты экспериментов}
\par Корректность алгоритма линейной фильтрации изображений с использованием ядра Гаусса 3x3 и горизонтального разбиения была проверена путем сравнения результатов фильтрации с эталонными изображениями.
\par Для проведения экспериментов по вычислению эффективности работы алгоритмов использовалась система со следующей конфигурацией:
\begin{itemize}
\item Процессор: AMD Ryzen 5 7600X, 6 ядер / 12 потоков
\item Оперативная память: 32 ГБ
\item Операционная система: Windows 10
\end{itemize}
\par Количество потоков для параллельных версий алгоритма было ограничено до 4.
\par В таблице приведены средние значения времени обработки изображения размером 1000x1000 пикселей за 10 запусков.

\par \textbf{Результаты экспериментов:}
\begin{center}
\begin{tabular}{ ||c | c | c || }
\hline Версия алгоритма & pipeline (сек) & task (сек)\\
\hline Последовательная & 2.1739988000 & 2.1785272000 \\
\hline TBB & 1.1405654000 & 1.0877458000 \\
\hline STL & 1.1593404000 & 1.1392744000 \\
\hline
\end{tabular}\\[3mm]
\end{center}

\newpage

% Выводы из результатов
\section*{Выводы из результатов}
\addcontentsline{toc}{section}{Выводы из результатов}
\par Для оценки эффективности параллельных алгоритмов линейной фильтрации изображений с использованием ядра Гаусса 3x3, были рассчитаны коэффициенты ускорения и эффективности. Расчеты основаны на среднем времени выполнения алгоритмов для изображения размером 1000х1000 пикселей.

\par Коэффициент ускорения \( p_i \) для каждой версии алгоритма рассчитывается по формуле:
$$ p_i = \frac{T_{si}}{T_{pi}} $$
где \( T_{si} \) - время выполнения последовательного алгоритма, а \( T_{pi} \) - время выполнения параллельного алгоритма для \( i \)-й версии.

\par Эффективность \( e_i \) для каждой версии алгоритма рассчитывается по формуле:
$$ e_i = \frac{p_i}{N} $$
где \( N \) - количество потоков (в данном случае 4).

\par Используя данные из предыдущей таблицы, получаем следующие результаты для двух различных типов задач: pipeline и task.

\begin{center}
\begin{tabular}{ ||c | c | c ||}
 \hline
 \multicolumn{3}{| c |}{pipeline}\\
 \hline
 Версия & Ускорение & Эффективность\\
 \hline
 Последовательная & 1 & 0.25 \\
 TBB & \( \frac{2.1739988000}{1.1405654000} \approx 1,906 \) & \( \frac{1.96}{4} \approx 0.476 \) \\
 STL & \( \frac{2.1739988000}{1.1593404000} \approx 1,875 \) & \( \frac{1.89}{4} \approx 0.469 \) \\
 \hline
\end{tabular}
\end{center}

\vspace{2em}

\begin{center}
\begin{tabular}{ ||c | c | c ||}
 \hline
 \multicolumn{3}{| c |}{task}\\
 \hline
 Версия & Ускорение & Эффективность\\
 \hline
 Последовательная & 1 & 0.25 \\
 TBB & \( \frac{2.1785272000}{1.0877458000} \approx 2,003 \) & \( \frac{1.97}{4} \approx 0.5 \) \\
 STL & \( \frac{2.1785272000}{1.1392744000} \approx 1,912 \) & \( \frac{1.89}{4} \approx 0.478 \) \\
 \hline
\end{tabular}
\end{center}

\textbf{Выводы}:
\begin{itemize}
    \item В обоих типах задач (pipeline и task) наилучшее ускорение и эффективность показала версия TBB.
    \item Наименьшее ускорение и эффективность демонстрирует STL версия.
    \item Распараллеливание алгоритма привело к ускорению его выполнения примерно в два раза по сравнению с последовательной версией.
\end{itemize}

\newpage


% Заключение
\section*{Заключение}
\addcontentsline{toc}{section}{Заключение}
\par В рамках данной работы были рассмотрены теоретические основы и реализованы три версии алгоритма линейной фильтрации изображений с использованием ядра Гаусса 3x3 и вертикального разбиения: последовательная реализация, параллельная реализация с использованием технологии TBB и параллельная реализация с использованием стандартной библиотеки C++. Проведено сравнение реализаций и анализ производительности каждой версии.
\par Вертикальное разбиение изображения позволило существенно ускорить процесс фильтрации и повысить производительность, что подтверждается экспериментальными данными. Данная методика может быть полезна для разработчиков программного обеспечения в области обработки изображений и компьютерного зрения, а также для исследователей, занимающихся параллельными вычислениями и оптимизацией алгоритмов.

% Литература
\newpage
\begin{thebibliography}{1}
\addcontentsline{toc}{section}{Литература}
\bibitem{TBB} Intel Threading Building Blocks. URL: \url{https://software.intel.com/content/www/us/en/develop/tools/oneapi/components/threading-building-blocks.html}
\bibitem{C++ Standard Library} C++ Standard Library. URL: \url{https://en.cppreference.com/w/cpp/standard_library}
\item Лекции Сысоева А.В. по курсу "Параллельное программирование для систем с общей памятью".
\end{thebibliography}

\newpage

% Код программы
\section*{Код программы}
\addcontentsline{toc}{section}{Код программы}
\par Реализация алгоритма на языке программирования C++ представлена ниже.

\begin{lstlisting}
// Copyright 2024 Kochetov Nikolay
#include "seq/kochetov_n_gauss_filter/include/gauss_filter.hpp"

#include <gtest/gtest.h>

#include <random>
#include <vector>

int* GaussFilter::getImgId(int a, int b) { return &input[a * x + b]; }

int* GaussFilter::getResId(int a, int b) { return &result[a * x + b]; }

int GaussFilter::clampData(int value, int min, int max) {
  if (value < min) return min;
  if (value > max) return max;
  return value;
}

bool GaussFilter::pre_processing() {
  internal_order_test();

  x = taskData->inputs_count[0];
  y = taskData->inputs_count[1];

  input = reinterpret_cast<int*>(taskData->inputs[0]);
  result = reinterpret_cast<int*>(taskData->outputs[0]);

  return true;
}

bool GaussFilter::run() {
  internal_order_test();

  for (int i = 1; i < x - 1; ++i) {
    for (int j = 1; j < y - 1; ++j) {
      int sum = 0;
      for (int ki = -1; ki <= 1; ++ki) {
        for (int kj = -1; kj <= 1; ++kj) {
          sum += *getImgId(ki + i, kj + j) * kernel[ki + 1][kj + 1];
        }
      }
      *getResId(i, j) = (int)(sum / 16);
    }
  }
  return true;
}

bool GaussFilter::post_processing() {
  internal_order_test();

  for (int i = 0; i < x; ++i) {
    for (int j = 0; j < y; ++j) {
      *getResId(i, j) = clampData(*getResId(i, j), 0, 255);
    }
  }
  return true;
}

bool GaussFilter::validation() {
  internal_order_test();

  return !taskData->inputs.empty() && !taskData->outputs.empty() && !taskData->inputs_count.empty() &&
         !taskData->outputs_count.empty() && taskData->inputs[0] != nullptr && taskData->outputs[0] != nullptr &&
         taskData->outputs_count[1] == taskData->inputs_count[1] && taskData->outputs_count[0] >= 3 &&
         taskData->outputs_count[1] >= 3;
}
\end{lstlisting}
\par Отличие в TBB реализации
\begin{lstlisting}
bool GaussFilter::run() {
  internal_order_test();

  tbb::parallel_for(1, x - 1, [&](int i) {
    for (int j = 1; j < y - 1; ++j) {
      int sum = 0;
      for (int ki = -1; ki <= 1; ++ki) {
        for (int kj = -1; kj <= 1; ++kj) {
          sum += *getImgId(ki + i, kj + j) * kernel[ki + 1][kj + 1];
        }
      }
      *getResId(i, j) = (int)(sum / 16);
    }
  });
  return true;
}
\end{lstlisting}
\par Отличие в STL реализации
\begin{lstlisting}
void GaussFilter::applyFilter(int startRow, int endRow) {
  for (int i = startRow; i < endRow; ++i) {
    for (int j = 1; j < y - 1; ++j) {
      int sum = 0;
      for (int ki = -1; ki <= 1; ++ki) {
        for (int kj = -1; kj <= 1; ++kj) {
          sum += *getImgId(ki + i, kj + j) * kernel[ki + 1][kj + 1];
        }
      }
      *getResId(i, j) = (int)(sum / 16);
    }
  }
}

bool GaussFilter::run() {
  internal_order_test();

  const int numThreads = std::thread::hardware_concurrency();
  std::vector<std::thread> threads;
  int chunkSize = (x - 2) / numThreads;

  for (int i = 0; i < numThreads; ++i) {
    int startRow = 1 + i * chunkSize;
    int endRow = (i == numThreads - 1) ? x - 1 : startRow + chunkSize;
    threads.emplace_back(&GaussFilter::applyFilter, this, startRow, endRow);
  }

  for (auto& t : threads) {
    t.join();
  }

  return true;
}
\end{lstlisting}


\end{document}
