\documentclass[]{article}

\usepackage[warn]{mathtext}
\usepackage[T2A]{fontenc}
\usepackage[utf8]{luainputenc}
\usepackage[english, russian]{babel}
\usepackage[pdfpagemode=UseNone,colorlinks,allcolors=black]{hyperref}
\usepackage{breakurl}
\usepackage[12pt]{extsizes}
\usepackage{color}
\usepackage{geometry}
\usepackage{enumitem}
\usepackage{multirow}
\usepackage{graphicx}
\usepackage{indentfirst}
\usepackage{amsmath}
\usepackage{amssymb}
\usepackage{listings}

\geometry{a4paper,top=2cm,bottom=2cm,left=3cm,right=1.5cm}
\setlength{\parskip}{0.5cm}
\setlist{nolistsep, itemsep=0.3cm,parsep=0pt}

\makeatletter
\renewcommand\@biblabel[1]{#1.\hfil}
\makeatother

\usepackage{amsthm}

\theoremstyle{remark}
\newtheorem*{remark}{Примечание}

\theoremstyle{definition}
\newtheorem*{defi}{Определение}

\usepackage{titlesec}
\titlelabel{\thetitle.\quad}

\newcommand{\term}[1]{$\mathtt{#1}$}
\newcommand{\termn}[1]{\mathtt{#1}}

\newcommand{\cpp}{\textit{}}

\begin{document}

\begin{titlepage}

\begin{center}
\MakeUppercase{Министерство науки и высшего образования Российской~Федерации}
\end{center}

\begin{center}
Федеральное государственное автономное образовательное учреждение высшего~образования \\
\textbf{Национальный исследовательский Нижегородский государственный университет им.~Н.И. Лобачевского}
\end{center}

\begin{center}
\textbf{Институт информационных технологий, математики и механики}
\end{center}
\begin{center}
\textbf{Кафедра математического обеспечения и суперкомпьютерных технологий}
\end{center}
\begin{center}
Направление подготовки: <<Программная инженерия>>
\end{center}
\begin{center}
Профиль: <<Разработка программно-информационных систем>>
\end{center}

\vspace{3em}

\begin{center}
\textbf{\Large\MakeUppercase{Отчёт по лабораторной работе}} \\
\end{center}
\begin{center}
\textbf{\Large<<Вычисление многомерных интегралов с использованием многошаговой схемы (метод трапеций)>>} \\
\end{center}

\vspace{5em}

\newbox{\lbox}
\savebox{\lbox}{\hbox{text}}
\newlength{\maxl}
\setlength{\maxl}{\wd\lbox}
\hfill\parbox{7cm}{
\hspace*{5cm}\hspace*{-5cm}\textbf{Выполнил:} \\ студент группы 	3821Б1ПР2 \\ Жаткин В. А.\\
\\

\hspace*{5cm}\hspace*{-5cm}\textbf{Проверил:} \\ Нестеров А. Ю.
}
\vspace{\fill}

\begin{center} Нижний Новгород \\ 2024 \end{center}

\end{titlepage}

\setcounter{page}{2}

% Содержание
\tableofcontents
\newpage

\section{Введение}

\par При решении инженерных задач часто приходится вычислять значения определённого интеграла. В случаях, когда не удаётся выразить интеграл в замкнутой форме, или полученная формула сложна, или подынтегральная функция задана таблично, пользуются численным интегрированием. В основе численного интегрирования лежит приближенное вычисление площади под кривой, описываемой подынтегральной функцией. Наиболее простыми для реализации являются методы, для которых значения xзаданы с постоянным шагом. Как раз метод трапеций, короый будет расмотрен в этом отчёте один из таких 

\par Основной целью данной лабораторной работы является реализация и исследование эффективности оператора Собеля для выделения ребер на изображении. В работе будут рассмотрены несколько вариантов реализации, включая последовательную и параллельные версии с использованием OpenMP, Threading Building Blocks (TBB), и стандартной библиотеки шаблонов (STL).

\par Параллелизация алгоритмов обработки изображений позволяет значительно ускорить вычисления, что особенно важно при обработке больших изображений в реальном времени. В данной работе будут исследованы следующие аспекты:

\begin{itemize}
    \item Реализация метода трапеций для многомерных интегралов;
    \item Параллельная реализация с использованием OpenMP;
    \item Параллельная реализация с использованием TBB;
    \item Параллельная реализация с использованием STL.
\end{itemize}

\par Цель работы — исследовать и сравнить производительность всех реализаций оператора Собеля, провести эксперименты на различных изображениях и на разном количестве процессорных ядер, чтобы сделать выводы об эффективности параллелизации при использовании различных библиотек.

\par Экспериментальная часть включает замеры времени выполнения для каждой реализации и анализ полученных результатов, чтобы определить, как изменение параметров влияет на производительность. Это позволит выявить потенциальные преимущества и недостатки каждого подхода к параллелизации и обосновать выбор оптимального варианта для различных условий использования.

\newpage

\section{Описание алгоритмов}

\par В данной работе используются следующие алгоритмы для вычисления  многомерных интегралов.

\subsection{Метод трапеций}

\par Вычисление многомерных интегралов с использованием метода трапеций является расширением одномерного метода трапеций на случай нескольких переменных. Метод трапеций заключается в аппроксимации подынтегральной функции на каждом элементарном интервале прямой линией (или в многомерном случае гиперплоскостью) и последующем суммировании площадей (или объемов) полученных трапеций (или гипертрапеций).

\par Для вычисления многомерного интеграла методом трапеций можно использовать многошаговую схему, которая включает в себя разбиение области интегрирования на элементарные ячейки и применение метода трапеций к каждой из них.

\parВажно отметить, что точность метода трапеций повышается с уменьшением размеров ячеек разбиения, но при этом увеличивается количество вычислений. Для повышения точности можно использовать адаптивные схемы, которые позволяют уменьшать размеры ячеек в областях с бóльшими изменениями функции.

\subsection{Параллельная реализация с использованием OpenMP}

\par OpenMP позволяет распараллелить вычисления с использованием директив компилятора. Основные шаги параллельной реализации метода трапеций с использованием OpenMP:

\begin{enumerate}
    \item Разделение работы.
    \item Инициализация OpenMP.
    \item Использование директив OpenMP.
    \item Параллельное выполнение суммирования.
    \item Сбор результатов.
\end{enumerate}

\subsection{Параллельная реализация с использованием TBB}

\par TBB — это библиотека для параллельного программирования, предоставляющая высокоуровневые абстракции для работы с потоками. Основные шаги параллельной реализации метода трапеций с использованием TBB:

\begin{enumerate}
    \item Использование \texttt{tbb::parallel\_for} для управления параллельными вычислениями.
    \item Параллельное выполнение суммирования.
    \item Сбор результатов.
\end{enumerate}

\subsection{Параллельная реализация с использованием STL}

\par STL (Standard Template Library) предоставляет алгоритмы для параллельного выполнения с использованием стандартных функций. Основные шаги параллельной реализации метода трапеций с использованием STL:

\begin{enumerate}
    \item Использование \texttt{std::for\_each} с указанием \texttt{std::execution::par} для параллельного выполнения.
    \item Параллельное выполнение суммирования.
    \item Сбор результатов.
\end{enumerate}

\newpage

\section{Программная реализация}

\subsection{Последовательная реализация}

\par Последовательная реализация метода трапеций:

\begin{lstlisting}[language=C++]
double trapezoidal_integral(const std::function<double(double, double)>& f, double lowerx, double upperx, double lowery,
                            double uppery, int nx, int ny) {
  double hx = (upperx - lowerx) / nx;
  double hy = (uppery - lowery) / ny;
  double sum = 0.5 * (f(lowerx, lowery) + f(upperx, uppery));

  for (int i = 1; i < nx; ++i) {
    double x = lowerx + hx * i;
    for (int j = 1; j < ny; ++j) {
      double y = lowery + hy * j;
      sum += f(x, y);
    }
  }

  return hx * hy * sum;
}
\end{lstlisting}

\subsection{Параллельная реализация с использованием OpenMP}

\par Параллельная реализация метода трапеций с использованием OpenMP:

\begin{lstlisting}[language=C++]
double trapezoidal_integralOMP(const std::function<double(double, double)>& f, double lowerx, double upperx,
                               double lowery, double uppery, int nx, int ny) {
  double hx = (upperx - lowerx) / nx;
  double hy = (uppery - lowery) / ny;
  double sum = 0.5 * (f(lowerx, lowery) + f(upperx, uppery));

#pragma omp parallel for reduction(+ : sum)
  for (int i = 1; i < nx; ++i) {
    double x = lowerx + hx * i;
    double local_sum = 0.0;
    for (int j = 1; j < ny; ++j) {
      double y = lowery + hy * j;
      local_sum += f(x, y);
    }
    sum += local_sum;
  }

  return hx * hy * sum;
}

\end{lstlisting}

\subsection{Параллельная реализация с использованием TBB}

\par Параллельная реализация метода трапеций с использованием TBB:

\begin{lstlisting}[language=C++]
double trapezoidal_integralTBB(const std::function<double(double, double)>& f, double lowerx, double upperx,
                               double lowery, double uppery, int nx, int ny) {
  double hx = (upperx - lowerx) / nx;
  double hy = (uppery - lowery) / ny;
  double sum = 0.5 * (f(lowerx, lowery) + f(upperx, uppery));

  oneapi::tbb::combinable<double> localSums;

  oneapi::tbb::parallel_for(oneapi::tbb::blocked_range<int>(1, nx), [&](const oneapi::tbb::blocked_range<int>& range) {
    auto& localSum = localSums.local();
    for (int i = range.begin(); i < range.end(); ++i) {
      double x = lowerx + hx * i;
      for (int j = 1; j < ny; ++j) {
        double y = lowery + hy * j;
        localSum += f(x, y);
      }
    }
  });
  sum += localSums.combine(std::plus<>());

  return hx * hy * sum;
}
\end{lstlisting}

\subsection{Параллельная реализация с использованием STL}

\par Параллельная реализация метода трапеций с использованием STL:

\begin{lstlisting}[language=C++]
double trapezoidal_integralSTL(const std::function<double(double, double)>& f, double lowerx, double upperx,
                               double lowery, double uppery, int nx, int ny) {
  double hx = (upperx - lowerx) / nx;
  double hy = (uppery - lowery) / ny;
  double sum = 0.5 * (f(lowerx, lowery) + f(upperx, uppery));

  std::vector<std::future<double>> futures(nx - 1);

  for (int i = 1; i < nx; ++i) {
    futures[i - 1] = std::async(
        std::launch::async,
        [&](int ii) {
          double x = lowerx + hx * ii;
          double inner_sum = 0.0;
          for (int j = 1; j < ny; ++j) {
            double y = lowery + hy * j;
            inner_sum += f(x, y);
          }
          return inner_sum;
        },
        i);
  }

  for (auto& future : futures) {
    sum += future.get();
  }

  return hx * hy * sum;
}
\end{lstlisting}

\newpage

\section{Экспериментальная часть}

\subsection{Условия проведения экспериментов}

\par Для проведения экспериментов использовались следующие условия:

\begin{itemize}
    \item Компьютер с процессором Intel Core i7700K и 16 ГБ оперативной памяти;
    \item Операционная система Windows 10(сборка);
    \item Компилятор Visual Studio 17 2022;
    \item Будут подставлены в одну и ту же функцию разное количество шагов
\end{itemize}

\subsection{Результаты и анализ}

\par В таблице 1 приведены результаты измерения времени выполнения различных реализаций метода трапеций с разным количеством шагов.

\begin{table}[h]
    \centering
    \begin{tabular}{|c|c|c|c|c|}
        \hline
        Количство шагов по x и по y & Последовательная реализация & OpenMP & TBB & STL \\
        \hline
        50x50 & 0.06 с & 0.03 с & 0.03 с & 0.03 с \\
        \hline
        100x100 & 0.24 с & 0.11 с & 0.11 с & 0.11 с \\
        \hline
        200x200 & 0.76 с & 0.42 с & 0.42 с & 0.42 с \\
        \hline
    \end{tabular}
    \caption{Время выполнения метода трапеций с разным количеством шагов}
    \label{tab:results}
\end{table}

\par Как видно из таблицы, параллельные реализации с использованием OpenMP, TBB и STL показывают схожие результаты и значительно превосходят последовательную реализацию по времени выполнения. Это подтверждает эффективность параллелизации для ускорения вычислений при обработке изображений.

\newpage

\section{Заключение}

\par В представленном исследовании были разработаны и изучены различные версии метода трапеций. Рассматривались как традиционная последовательная версия, так и параллельные версии с применением технологий OpenMP, TBB и STL.

\par Результаты тестирования свидетельствуют о том, что параллельные версии существенно превосходят последовательную по скорости выполнения, что подчеркивает эффективность использования параллельных вычислений для задач, связанных с обработкой изображений. Реализации с использованием OpenMP, TBB и STL показали аналогичные результаты, что дает возможность выбора наиболее подходящего инструмента на основе личных предпочтений и специфических требований проекта.

\par В будущих исследованиях можно расширить область изучения, включив в анализ другие подходы к параллелизации и оптимизации, а также применить более сложные операторы для решения задач в области компьютерного зрения.

\begin{thebibliography}{9}

\bibitem{Метод рапеций}
Блюмин А.Г., Федотов А.А., Храпов П.В. Численные методы вычисления интегралов и решение задач для обыкновенных дифференциальных уравнений.

\bibitem{OpenMP}
OpenMP Architecture Review Board. (2021). OpenMP Application Programming Interface Version 5.1. \url{https://www.openmp.org/specifications/}

\bibitem{TBB}
Reinders, J. (2007). Intel Threading Building Blocks: Outfitting C++ for Multi-core Processor Parallelism. O'Reilly Media.

\bibitem{STL}
ISO/IEC. (2017). ISO/IEC 14882:2017: Programming Languages - C++. International Organization for Standardization.

\end{thebibliography}

\newpage

\section{Приложение}

\begin{lstlisting}[language=C++]
// Copyright 2024 Zhatkin Vyacheslav
#pragma once

#include <functional>
#include <memory>
#include <string>
#include <utility>
#include <vector>

#include "core/task/include/task.hpp"

class ZhatkinTaskSequential : public ppc::core::Task {
 public:
  explicit ZhatkinTaskSequential(std::shared_ptr<ppc::core::TaskData> taskData_,
                                 std::function<double(double, double)> func)
      : Task(std::move(taskData_)), f(std::move(func)) {}
  bool pre_processing() override;
  bool validation() override;
  bool run() override;
  bool post_processing() override;

 private:
  std::function<double(double, double)> f;
  double lowerx{}, upperx{}, lowery{}, uppery{}, res = 0.0;
  int nx{}, ny{};
};



double trapezoidal_integral(const std::function<double(double, double)>& f, double lowerx, double upperx, double lowery,
                            double uppery, int nx, int ny);

                            

double trapezoidal_integralOMP(const std::function<double(double, double)>& f, double lowerx, double upperx, double lowery,
                            double uppery, int nx, int ny);



double trapezoidal_integralTBB(const std::function<double(double, double)>& f, double lowerx, double upperx, double lowery,
                            double uppery, int nx, int ny);



double trapezoidal_integralSTL(const std::function<double(double, double)>& f, double lowerx, double upperx, double lowery,
                            double uppery, int nx, int ny);





#include "seq/zhatkin_v_the_trapezoid_method/include/trapezoid_method.hpp"

double trapezoidal_integral(const std::function<double(double, double)>& f, double lowerx, double upperx, double lowery,
                            double uppery, int nx, int ny) {
  double hx = (upperx - lowerx) / nx;
  double hy = (uppery - lowery) / ny;
  double sum = 0.5 * (f(lowerx, lowery) + f(upperx, uppery));

  for (int i = 1; i < nx; ++i) {
    double x = lowerx + hx * i;
    for (int j = 1; j < ny; ++j) {
      double y = lowery + hy * j;
      sum += f(x, y);
    }
  }

  return hx * hy * sum;
}



double trapezoidal_integralOMP(const std::function<double(double, double)>& f, double lowerx, double upperx,
                               double lowery, double uppery, int nx, int ny) {
  double hx = (upperx - lowerx) / nx;
  double hy = (uppery - lowery) / ny;
  double sum = 0.5 * (f(lowerx, lowery) + f(upperx, uppery));

#pragma omp parallel for reduction(+ : sum)
  for (int i = 1; i < nx; ++i) {
    double x = lowerx + hx * i;
    double local_sum = 0.0;
    for (int j = 1; j < ny; ++j) {
      double y = lowery + hy * j;
      local_sum += f(x, y);
    }
    sum += local_sum;
  }

  return hx * hy * sum;
}




double trapezoidal_integralTBB(const std::function<double(double, double)>& f, double lowerx, double upperx,
                               double lowery, double uppery, int nx, int ny) {
  double hx = (upperx - lowerx) / nx;
  double hy = (uppery - lowery) / ny;
  double sum = 0.5 * (f(lowerx, lowery) + f(upperx, uppery));

  oneapi::tbb::combinable<double> localSums;

  oneapi::tbb::parallel_for(oneapi::tbb::blocked_range<int>(1, nx), [&](const oneapi::tbb::blocked_range<int>& range) {
    auto& localSum = localSums.local();
    for (int i = range.begin(); i < range.end(); ++i) {
      double x = lowerx + hx * i;
      for (int j = 1; j < ny; ++j) {
        double y = lowery + hy * j;
        localSum += f(x, y);
      }
    }
  });
  sum += localSums.combine(std::plus<>());

  return hx * hy * sum;
}




double trapezoidal_integralSTL(const std::function<double(double, double)>& f, double lowerx, double upperx,
                               double lowery, double uppery, int nx, int ny) {
  double hx = (upperx - lowerx) / nx;
  double hy = (uppery - lowery) / ny;
  double sum = 0.5 * (f(lowerx, lowery) + f(upperx, uppery));

  std::vector<std::future<double>> futures(nx - 1);

  for (int i = 1; i < nx; ++i) {
    futures[i - 1] = std::async(
        std::launch::async,
        [&](int ii) {
          double x = lowerx + hx * ii;
          double inner_sum = 0.0;
          for (int j = 1; j < ny; ++j) {
            double y = lowery + hy * j;
            inner_sum += f(x, y);
          }
          return inner_sum;
        },
        i);
  }

  for (auto& future : futures) {
    sum += future.get();
  }

  return hx * hy * sum;
}



bool ZhatkinTaskSequential::pre_processing() {
  internal_order_test();

  try {
    lowerx = reinterpret_cast<double*>(taskData->inputs[0])[0];
    upperx = reinterpret_cast<double*>(taskData->inputs[0])[1];
    lowery = reinterpret_cast<double*>(taskData->inputs[0])[2];
    uppery = reinterpret_cast<double*>(taskData->inputs[0])[3];
    nx = reinterpret_cast<int*>(taskData->inputs[1])[0];
    ny = reinterpret_cast<int*>(taskData->inputs[1])[1];
  } catch (...) {
    return false;
  }

  return true;
}




bool ZhatkinTaskSequential::post_processing() {
  internal_order_test();
  reinterpret_cast<double*>(taskData->outputs[0])[0] = res;
  return true;
}




bool ZhatkinTaskSequential::validation() {
  internal_order_test();
  return taskData->inputs_count[0] == 4 && taskData->inputs_count[1] == 2 && taskData->outputs_count[0] == 1;
}




bool ZhatkinTaskSequential::run() {
  try {
    internal_order_test();
    res = trapezoidal_integral(f, lowerx, upperx, lowery, uppery, nx, ny);
  } catch (...) {
    return false;
  }
  return true;
}
                            
\end{lstlisting}
\end{document}
