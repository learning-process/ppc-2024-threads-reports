\documentclass{report}

\usepackage[T2A]{fontenc}
\usepackage[utf8]{luainputenc}
\usepackage[english, russian]{babel}
\usepackage{tempora}
\usepackage[12pt]{extsizes}
\usepackage{listings}
\usepackage{color}
\usepackage{geometry}
\usepackage{enumitem}
\usepackage{graphicx}
\usepackage{indentfirst}
\usepackage{amsmath}
\usepackage{soul}
\usepackage{hyperref}

\sethlcolor{gray}

\geometry{a4paper,top=2cm,bottom=2cm,left=2.5cm,right=1.5cm}
\setlength{\parskip}{0.5cm}
\setlist{nolistsep, itemsep=0.3cm,parsep=0pt}

\lstset{language=C++,
        basicstyle=\footnotesize,
		keywordstyle=\color{blue}\ttfamily,
		stringstyle=\color{red}\ttfamily,
		commentstyle=\color{green}\ttfamily,
		morecomment=[l][\color{red}]{\#},
		tabsize=4,
		breaklines=true,
  		breakatwhitespace=true,
  		title=\lstname,
}

\begin{document}

\begin{titlepage}

\begin{center}
Министерство науки и высшего образования Российской Федерации
\end{center}

\begin{center}
Федеральное государственное автономное образовательное учреждение высшего образования \\
Национальный исследовательский Нижегородский государственный университет имени Н.И. Лобачевского
\end{center}

\begin{center}
Институт информационных технологий, математики и механики \\
Кафедра математического обеспечения и суперкомпьютерных технологий \\
Направление подготовки: Программная инженерия
\end{center}

\vspace{4em}

\begin{center}
\textbf{\Large Отчет по лабораторной работе \\
\vspace{0.5em}
<<Сортировка Хоара с простым слиянием>>} \\
\end{center}

\vspace{4em}

\hfill\parbox{7cm}{
\hspace*{5cm}\hspace*{-5cm}\textbf{Выполнил:} \\ студент группы 3821Б1ПР1\\
Подъячих М. В.\\
\\
\hspace*{5cm}\hspace*{-5cm}\textbf{Проверил:}\\ аспирант\\Нестеров А. Ю.\\
}

\vspace{\fill}

\begin{center} Нижний Новгород \\ 2024 \end{center}

\end{titlepage}

\setcounter{page}{2}

\tableofcontents
\newpage

\section*{Введение}
\addcontentsline{toc}{section}{Введение}
\par В данной лабораторной работе рассматривается сортировка Хоара (Quicksort) с простым слиянием, а также её параллельные реализации с использованием технологий OpenMP, TBB и STL threads. Целью является сравнение производительности различных реализаций на многопоточных системах.

\newpage

\section*{Постановка задачи}
\addcontentsline{toc}{section}{Постановка задачи}
\par \textbf{Цели и задачи:}
\begin{enumerate}
    \item Реализовать алгоритм сортировки Хоара (Quicksort) с простым слиянием.
    \item Реализовать параллельные версии алгоритма с использованием OpenMP, TBB и STL threads.
    \item Провести сравнительный анализ производительности различных реализаций.
\end{enumerate}

\newpage

\section*{Описание алгоритма}
\addcontentsline{toc}{section}{Описание алгоритма}
\par Сортировка Хоара, или Quicksort, является одним из наиболее эффективных алгоритмов сортировки. Основная идея заключается в выборе опорного элемента (pivot) и разделении массива на две части: элементы меньше и больше опорного. Затем рекурсивно сортируются обе части.

\par Пример псевдокода:
\begin{lstlisting}
algorithm quicksort(A, lo, hi) is
   if lo < hi then
       p:= partition(A, lo, hi)
       quicksort(A, lo, p)
       quicksort(A, p + 1, hi)
algorithm partition(A, low, high) is
   pivot:= A[(low + high) / 2]
   i:= low
   j:= high
   loop forever
       while A[i] < pivot 
              i:= i + 1
       while A[j] > pivot
              j:= j - 1
       if i >= j then
           return j
       swap A[i++] with A[j--]
\end{lstlisting}

\newpage

\section*{Описание параллельных версий}
\addcontentsline{toc}{section}{Описание параллельных версий}
\par Для распараллеливания алгоритма сортировки Хоара были использованы следующие технологии: OpenMP, TBB и STL threads. Основная идея заключалась в распараллеливании рекурсивных вызовов для сортировки частей массива.

\par Однако распараллеливание каждого вызова функции сортировки на подотрезке приводило к увеличению времени исполнения, так как создвалось много добалнительных потоков на маленькие подотрезки массива. Было выбрано ограничить применение параллелизма параметром max\_segment\_length. Который лучше всего себя показывает на значениях около \(\frac{n}{N}\), где \(n\) - размер сортируемого массива, а \(N\) - количесво доступных процессоров.

\par \textbf{Версия с OpenMP:}
\par Используется набор директив, позволяющих параллельно выполнять сортировку в каждом блоке:
\begin{enumerate}
    \item \#pragma omp parallel sections
    \item \#pragma omp section 
\end{enumerate}
\begin{lstlisting}
#pragma omp parallel sections
{
    #pragma omp section
    {
        omp_hoare_sort(left, j);
    }
    #pragma omp section
    {
        omp_hoare_sort(j + 1, right);
    }
}
\end{lstlisting}

\par \textbf{Версия с TBB:}
\par Используется функция tbb::parallel\_invoke(), которая выполняет переданные функции параллельно:
\begin{lstlisting}
tbb::parallel_invoke(
    [&] { tbb_hoare_sort(left, j); }, 
    [&] { tbb_hoare_sort(j + 1, right); }
);
\end{lstlisting}

\par \textbf{Версия с STL threads:}
\par Используются потоки из стандартной библиотеки C++:
\begin{enumerate}
    \item \texttt{std::thread} - для создания потоков
    \item \texttt{std::thread::join} - для ожидания завершения выполнения потоков
\end{enumerate}
\begin{lstlisting}
std::thread t1([&]{ stl_hoare_sort(left, j); });
std::thread t2([&]{ stl_hoare_sort(j + 1, right); });
t1.join();
t2.join();
\end{lstlisting}

\newpage

\section*{Результаты экспериментов}
\addcontentsline{toc}{section}{Результаты экспериментов}
\par Для проведения экспериментов использовалась система со следующими характеристиками:
\begin{itemize}
    \item Процессор: Intel Core i5-12400, 12 ядер
    \item Оперативная память: 32 ГБ
    \item Операционная система: Windows 10
\end{itemize}

\par \textbf{Данные:}
\par Для тестирования использовался случайно сгенерированный массив размером \(2000000\) элементов. Каждый элемент массива являлся случайным целым числом в диапазоне от \(-1e9\) до \(1e9\). Время исполнения усреднено по 10 последовательным запускам
\par 

\par \textbf{Результаты:}
\begin{center}
\begin{tabular}{ |c|c|c|c| }
    \hline
    Версия алгоритма & Время выполнения (мс) & Ускорение & Эффективность \\ 
    \hline
    Последовательная & 856 & 1.0 & 0.083 \\ 
    \hline
    OpenMP & 301 & 2.84 & 0.237 \\ 
    \hline
    TBB & 279 & 3.07 & 0.256 \\ 
    \hline
    STL threads & 263 & 3.25 & 0.271 \\ 
    \hline
\end{tabular}
\end{center}

\par \textbf{Вычисление прироста скорости:}
\par Прирост скорости для каждой параллельной версии вычислялся как отношение времени выполнения последовательной версии к времени выполнения параллельной версии:
\[ \text{Ускорение} = \frac{T_{\text{посл.}}}{T_{\text{пар.}}} \]
\par Эффективность для каждой параллельной версии вычислялся как отношение ускорения параллельной версии к количеству используемых процессоров:
\[ \text{Эффективность} = \frac{{\text{Ускорение}}}{{\text{Число процессоров}}} \]

\newpage

\section*{Заключение}
\addcontentsline{toc}{section}{Заключение}
\par В ходе лабораторной работы были реализованы последовательная и параллельные версии алгоритма сортировки Хоара. Проведенный анализ показал, что параллельные версии значительно превосходят последовательную по производительности, особенно версии с использованием OpenMP и TBB. На тестовом наборе данных размером 2000000 элементов параллельные версии обеспечили ускорение более чем в два раза. Увеличение скорости с ростом числа процессоров показывает эффективность распараллеливания, однако прирост становится менее значительным при увеличении числа ядер из-за дополнительных накладных расходов.

\newpage

\section*{Литература}
\addcontentsline{toc}{section}{Литература}
\begin{enumerate}
    \item Кнут Д.Э. Искусство программирования. Том 3. Сортировка и поиск. М.: Вильямс, 2002. — 830 с.
    \item OpenMP API Specification for Parallel Programming: \url{https://www.openmp.org}
    \item Intel Threading Building Blocks: \url{https://www.threadingbuildingblocks.org}
    \item C++ Standard Library: \url{https://en.cppreference.com/w/cpp/thread}
\end{enumerate}

\newpage

\section*{Приложение}
\addcontentsline{toc}{section}{Приложение}

\begin{lstlisting}{Последовательная версия}
// Copyright 2024 Podyachikh Mikhail

#include "seq/podyachikh_m_hoare_sort_simple_merge/include/hoare_sort.hpp"

#include <cassert>

using namespace std::chrono_literals;
using namespace Podyachikh;

bool HoareSort::pre_processing() {
  internal_order_test();
  try {
    // Init value for input and output
    _data = reinterpret_cast<vec_t *>(taskData->inputs[0])[0];
    if (taskData->inputs.size() == 2) {
      _comp = reinterpret_cast<compare_t *>(taskData->inputs[1])[0];
    } else {
      _comp = std::less<>();
    }
  } catch (...) {
    std::cout << "Failed pre_processing\n" << std::endl;
    return false;
  }
  return true;
}

bool HoareSort::validation() {
  internal_order_test();
  // Check count elements of output
  if (taskData->inputs.empty()) return false;
  if (taskData->inputs.size() > 2) return false;
  return taskData->outputs.size() == 1;
}

bool HoareSort::run() {
  internal_order_test();
  try {
    seq_hoare_sort(0, _data.size() - 1);
  } catch (...) {
    std::cout << "Run failure\n" << std::endl;
    return false;
  }
  return true;
}

bool HoareSort::post_processing() {
  internal_order_test();
  auto &res = reinterpret_cast<vec_t *>(taskData->outputs[0])[0];
  for (size_t i = 0; i < _data.size(); i++) {
    res[i] = _data[i];
  }
  return true;
}

int HoareSort::partition(int left, int right) {
  int pivot = _data[(left + right) / 2];
  left--;
  right++;
  while (true) {
    while (_comp(_data[++left], pivot))
      ;
    while (_comp(pivot, _data[--right]))
      ;
    if (left >= right) {
      return right;
    }
    std::swap(_data[left], _data[right]);
  }
}

void HoareSort::seq_hoare_sort(int left, int right) {
  if (left >= right) return;
  int mid = partition(left, right);
  seq_hoare_sort(left, mid);
  seq_hoare_sort(mid + 1, right);
}
\end{lstlisting}
\newpage

\begin{lstlisting}{OpenMP версия}
// Copyright 2024 Podyachikh Mikhail

#include "omp/podyachikh_m_hoare_sort_simple_merge/include/hoare_sort.hpp"

#include "omp.h"

using namespace std::chrono_literals;
using namespace PodyachikhOMP;

bool HoareSort::pre_processing() {
  internal_order_test();
  try {
    // Init value for input and output
    _data = reinterpret_cast<vec_t *>(taskData->inputs[0])[0];
    if (taskData->inputs.size() == 2) {
      _comp = reinterpret_cast<compare_t *>(taskData->inputs[1])[0];
    } else {
      _comp = std::less<>();
    }
  } catch (...) {
    std::cout << "Failed pre_processing\n" << std::endl;
    return false;
  }
  return true;
}

bool HoareSort::validation() {
  internal_order_test();
  // Check count elements of output
  if (taskData->inputs.empty()) return false;
  if (taskData->inputs.size() > 2) return false;
  return taskData->outputs.size() == 1;
}

bool HoareSort::run() {
  internal_order_test();
  omp_set_num_threads(4);
  max_segment_length = _data.size() / omp_get_max_threads();
  try {
    omp_hoare_sort(0, _data.size() - 1);
  } catch (...) {
    std::cout << "Run failure\n" << std::endl;
    return false;
  }
  return true;
}

bool HoareSort::post_processing() {
  internal_order_test();
  auto &res = reinterpret_cast<vec_t *>(taskData->outputs[0])[0];
  for (size_t i = 0; i < _data.size(); i++) {
    res[i] = _data[i];
  }
  return true;
}

void HoareSort::seq_hoare_sort(int left, int right) {
  if (left >= right) return;
  int pivot = _data[(left + right) / 2];
  int i = left - 1;
  int j = right + 1;
  while (true) {
    do {
      i++;
    } while (_comp(_data[i], pivot));
    do {
      j--;
    } while (_comp(pivot, _data[j]));
    if (i >= j) {
      break;
    }
    std::swap(_data[i], _data[j]);
  }
  seq_hoare_sort(left, j);
  seq_hoare_sort(j + 1, right);
}

void HoareSort::omp_hoare_sort(int left, int right) {
  if (left >= right) return;
  if (right - left < max_segment_length) {
    return seq_hoare_sort(left, right);
  }
  int pivot = _data[(left + right) / 2];
  int i = left - 1;
  int j = right + 1;
  while (true) {
    do {
      i++;
    } while (_comp(_data[i], pivot));
    do {
      j--;
    } while (_comp(pivot, _data[j]));
    if (i >= j) {
      break;
    }
    std::swap(_data[i], _data[j]);
  }

#pragma omp parallel sections default(shared)
  {
#pragma omp section
    { omp_hoare_sort(left, j); }
#pragma omp section
    { omp_hoare_sort(j + 1, right); }
  }
}
\end{lstlisting}
\newpage

\begin{lstlisting}{TBB версия}
// Copyright 2024 Podyachikh Mikhail

#include "tbb/podyachikh_m_hoare_sort_simple_merge/include/hoare_sort.hpp"

#include <oneapi/tbb.h>

using namespace std::chrono_literals;
using namespace PodyachikhTBB;

bool HoareSort::pre_processing() {
  internal_order_test();
  try {
    // Init value for input and output
    _data = reinterpret_cast<vec_t *>(taskData->inputs[0])[0];
    if (taskData->inputs.size() == 2) {
      _comp = reinterpret_cast<compare_t *>(taskData->inputs[1])[0];
    } else {
      _comp = std::less<>();
    }
  } catch (...) {
    std::cout << "Failed pre_processing\n" << std::endl;
    return false;
  }
  return true;
}

bool HoareSort::validation() {
  internal_order_test();
  // Check count elements of output
  if (taskData->inputs.empty()) return false;
  if (taskData->inputs.size() > 2) return false;
  return taskData->outputs.size() == 1;
}

bool HoareSort::run() {
  internal_order_test();
  max_segment_length = _data.size() / 12;
  try {
    tbb_hoare_sort(0, _data.size() - 1);
  } catch (...) {
    std::cout << "Run failure\n" << std::endl;
    return false;
  }
  return true;
}

bool HoareSort::post_processing() {
  internal_order_test();
  auto &res = reinterpret_cast<vec_t *>(taskData->outputs[0])[0];
  for (size_t i = 0; i < _data.size(); i++) {
    res[i] = _data[i];
  }
  return true;
}

void HoareSort::seq_hoare_sort(int left, int right) {
  if (left >= right) return;
  int pivot = _data[(left + right) / 2];
  int i = left - 1;
  int j = right + 1;
  while (true) {
    do {
      i++;
    } while (_comp(_data[i], pivot));
    do {
      j--;
    } while (_comp(pivot, _data[j]));
    if (i >= j) {
      break;
    }
    std::swap(_data[i], _data[j]);
  }
  seq_hoare_sort(left, j);
  seq_hoare_sort(j + 1, right);
}

void HoareSort::tbb_hoare_sort(int left, int right) {
  if (left >= right) return;
  if (right - left < max_segment_length) {
    return seq_hoare_sort(left, right);
  }
  int pivot = _data[(left + right) / 2];
  int i = left - 1;
  int j = right + 1;
  while (true) {
    do {
      i++;
    } while (_comp(_data[i], pivot));
    do {
      j--;
    } while (_comp(pivot, _data[j]));
    if (i >= j) {
      break;
    }
    std::swap(_data[i], _data[j]);
  }
  tbb::parallel_invoke([&] { tbb_hoare_sort(left, i); }, [&] { tbb_hoare_sort(j + 1, right); });
}
\end{lstlisting}
\newpage

\begin{lstlisting}{STL threads версия}
// Copyright 2024 Podyachikh Mikhail

#include "stl/podyachikh_m_hoare_sort_simple_merge/include/hoare_sort.hpp"

using namespace std::chrono_literals;
using namespace PodyachikhSTL;

bool HoareSort::pre_processing() {
  internal_order_test();
  try {
    // Init value for input and output
    _data = reinterpret_cast<vec_t *>(taskData->inputs[0])[0];
    if (taskData->inputs.size() == 2) {
      _comp = reinterpret_cast<compare_t *>(taskData->inputs[1])[0];
    } else {
      _comp = std::less<>();
    }
  } catch (...) {
    std::cout << "Failed pre_processing\n" << std::endl;
    return false;
  }
  return true;
}

bool HoareSort::validation() {
  internal_order_test();
  // Check count elements of output
  if (taskData->inputs.empty()) return false;
  if (taskData->inputs.size() > 2) return false;
  return taskData->outputs.size() == 1;
}

bool HoareSort::run() {
  internal_order_test();
  max_segment_length = _data.size() / 4;
  try {
    stl_hoare_sort(0, _data.size() - 1);
  } catch (...) {
    std::cout << "Run failure\n" << std::endl;
    return false;
  }
  return true;
}

bool HoareSort::post_processing() {
  internal_order_test();
  auto &res = reinterpret_cast<vec_t *>(taskData->outputs[0])[0];
  for (size_t i = 0; i < _data.size(); i++) {
    res[i] = _data[i];
  }
  return true;
}

void HoareSort::seq_hoare_sort(int left, int right) {
  if (left >= right) return;
  int pivot = _data[(left + right) / 2];
  int i = left - 1;
  int j = right + 1;
  while (true) {
    do {
      i++;
    } while (_comp(_data[i], pivot));
    do {
      j--;
    } while (_comp(pivot, _data[j]));
    if (i >= j) {
      break;
    }
    std::swap(_data[i], _data[j]);
  }
  seq_hoare_sort(left, j);
  seq_hoare_sort(j + 1, right);
}

void HoareSort::stl_hoare_sort(int left, int right) {
  if (left >= right) return;
  if (right - left < max_segment_length) {
    return seq_hoare_sort(left, right);
  }
  int pivot = _data[(left + right) / 2];
  int i = left - 1;
  int j = right + 1;
  while (true) {
    do {
      i++;
    } while (_comp(_data[i], pivot));
    do {
      j--;
    } while (_comp(pivot, _data[j]));
    if (i >= j) {
      break;
    }
    std::swap(_data[i], _data[j]);
  }

  std::thread lp_thread(&HoareSort::stl_hoare_sort, this, left, j);
  std::thread rp_thread(&HoareSort::stl_hoare_sort, this, j + 1, right);

  lp_thread.join();
  rp_thread.join();
}
\end{lstlisting}

\end{document}
