\documentclass{report}

\usepackage[warn]{mathtext}
\usepackage[T2A]{fontenc}
\usepackage[utf8]{luainputenc}
\usepackage[english, russian]{babel}
\usepackage[pdfpagemode=UseNone,colorlinks,allcolors=black]{hyperref}
\usepackage{tempora}
\usepackage[12pt]{extsizes}
\usepackage{listings}
\usepackage{color}
\usepackage{geometry}
\usepackage{enumitem}
\usepackage{multirow}
\usepackage{graphicx}
\usepackage{indentfirst}
\usepackage{amsmath}
\usepackage{soul}

\sethlcolor{gray}

\geometry{a4paper,top=2cm,bottom=2cm,left=2.5cm,right=1.5cm}
\setlength{\parskip}{0.5cm}
\setlist{nolistsep, itemsep=0.3cm,parsep=0pt}

\usepackage{listings}
\lstset{language=C++,
        basicstyle=\footnotesize,
		keywordstyle=\color{blue}\ttfamily,
		stringstyle=\color{red}\ttfamily,
		commentstyle=\color{green}\ttfamily,
		morecomment=[l][\color{red}]{\#}, 
		tabsize=4,
		breaklines=true,
  		breakatwhitespace=true,
  		title=\lstname,       
}

\makeatletter
\renewcommand\@biblabel[1]{#1.\hfil}
\makeatother

\begin{document}

\begin{titlepage}

\begin{center}
Министерство науки и высшего образования Российской Федерации
\end{center}

\begin{center}
Федеральное государственное автономное образовательное учреждение высшего образования \\
Национальный исследовательский Нижегородский государственный университет \newline им. Н.И. Лобачевского
\end{center}

\begin{center}
Институт информационных технологий, математики и механики \\
Кафедра математического обеспечения и суперкомпьютерных технологий \\
Направление подготовки: Программная инженерия
\end{center}

\begin{center}
\textbf{Отчет по курсу \\
\vspace{0.5em}
<<Параллельное программирование для систем с общей памятью>>} \\
\end{center}

\vspace{4em}

\begin{center}
\textbf{\LargeТема \\
\vspace{0.5em}
<<Построение выпуклой оболочки – проход Грэхема>>} \\
\end{center}

\vspace{4em}

\newbox{\lbox}
\savebox{\lbox}{\hbox{text}}
\newlength{\maxl}
\setlength{\maxl}{\wd\lbox}
\hfill\parbox{7cm}{
\hspace*{5cm}\hspace*{-5cm}\textbf{Выполнил:} \\ студент группы 3821Б1ПР2\\Кистримова Е. Е.\\
\\
\hspace*{5cm}\hspace*{-5cm}\textbf{Проверил:}\\ аспирант\\Нестеров А. Ю.\\
}
\vspace{\fill}

\begin{center} Нижний Новгород \\ 2024 \end{center}

\end{titlepage}

\setcounter{page}{2}

% Содержание
\tableofcontents
\newpage

% Введение
\section*{Введение}
\addcontentsline{toc}{section}{Введение}
\par Построение выпуклой оболочки является одной из фундаментальных задач вычислительной геометрии, широко используемой в компьютерной графике, обработке изображений и географических информационных системах. Выпуклая оболочка множества точек — это наименьший выпуклый многоугольник, содержащий все точки данного множества. Один из эффективных алгоритмов построения выпуклой оболочки — проход Грэхема, который имеет временную сложность \(O(n \log n)\) и основывается на сортировке точек по полярному углу и последовательной обработке этих точек для формирования выпуклой оболочки.

\par Целью данной работы является реализация алгоритма построения выпуклой оболочки методом прохода Грэхема и анализ его эффективности при параллельном исполнении. В рамках данной работы будут рассмотрены теоретические основы алгоритма Грэхема, описана его последовательная и параллельная реализации, а также проведены эксперименты по оценке производительности на различных наборах данных и конфигурациях системы. Результаты исследования могут быть полезны для разработчиков и исследователей в области вычислительной геометрии и параллельных вычислений.

\newpage

% Постановка задачи
\section*{Постановка задачи}
\addcontentsline{toc}{section}{Постановка задачи}
\par \textbf{В рамках данного задания необходимо выполнить следующие задачи:}
\begin{enumerate}
\item Реализовать последовательную версию алгоритма построения выпуклой оболочки методом прохода Грэхема.
\item Реализовать параллельные версии алгоритма с использованием технологий OpenMP, TBB и стандартной библиотеки C++.
\item Сравнить производительность реализованных версий.
\item Провести анализ полученных данных.
\end{enumerate}
\newpage

% Описание алгоритма
\section*{Описание алгоритма}
\addcontentsline{toc}{section}{Описание алгоритма}
\par Алгоритм построения выпуклой оболочки методом прохода Грэхема включает следующие основные этапы:
\begin{enumerate}
\item Выбор начальной точки с минимальной координатой y (и с минимальной координатой x в случае совпадения координат y).
\item Сортировка остальных точек по полярному углу относительно начальной точки.
\item Инициализация стековой структуры для хранения вершин выпуклой оболочки.
\item Последовательная обработка отсортированных точек и обновление стековой структуры, включающей добавление и удаление вершин в зависимости от углов поворота (по часовой или против часовой стрелки).
\end{enumerate}

\par Параллельные версии алгоритма используют распараллеливание этапов сортировки и обработки точек.

\newpage

% Описание схемы распараллеливания
\section*{Описание схемы распараллеливания}
\addcontentsline{toc}{section}{Описание схемы распараллеливания}
\par \textbf{Алгоритм построения выпуклой оболочки методом прохода Грэхема распараллеливается следующим образом:}
\begin{enumerate}
\item Сортировка точек по полярному углу выполняется с использованием параллельных алгоритмов сортировки.
\item Обработка точек и обновление стековой структуры распараллеливается путем разделения множества точек на подмножества, обработка которых выполняется параллельно, с последующим объединением результатов.
\end{enumerate}

\par В реализации с использованием OpenMP применяется директива pragma omp parallel for для распараллеливания цикла сортировки и обработки точек. В реализации с использованием TBB применяется tbb::parallel sort и другие средства библиотеки TBB. В реализации с использованием стандартной библиотеки C++ применяются стандартные потоки (\texttt{std::thread}).

\newpage

% Описание OpenMP, TBB и STL версий
\section*{Описание OpenMP, TBB и STL версий}
\par В различных версиях алгоритма построения выпуклой оболочки методом прохода Грэхема, использующих параллельные вычисления, основные этапы распараллеливания остаются неизменными. Однако ключевое различие между версиями, такими как OpenMP, TBB и STL, заключается в предоставляемых ими возможностях и методах распараллеливания.
\par \textbf{Версии реализации:}
\begin{enumerate}
\item В OpenMP используются директивы:
\vspace{0.5em}
\begin{enumerate}
\item[1.1] \begin{verbatim}
#pragma omp parallel for
\end{verbatim}
для распараллеливания сортировки и обработки точек.
\end{enumerate}
\item В TBB используется:
\vspace{0.5em}
\begin{enumerate}
\item[2.1] \begin{verbatim}
tbb::parallel_sort
\end{verbatim}
для параллельной сортировки точек.
\end{enumerate}
\item В STL используется:
\vspace{0.5em}
\begin{enumerate}
\item[3.1] \begin{verbatim}
std::thread
\end{verbatim}
для создания потоков, выполняющих сортировку и обработку точек.
\item[3.2] \begin{verbatim}
std::sort(std::execution::par, ...)
\end{verbatim}
для параллельной сортировки точек.
\end{enumerate}
\end{enumerate}

\newpage

% Экспериментальные данные
\section*{Экспериментальные данные}
\addcontentsline{toc}{section}{Экспериментальные данные}
\par Для оценки производительности различных версий алгоритма построения выпуклой оболочки методом прохода Грэхема были проведены эксперименты с различными наборами данных. Эксперименты выполнялись на компьютере с процессором AMD Ryzen 6800H и 16 ГБ оперативной памяти. В качестве компилятора использовался VS Code
\par \textbf{Результаты экспериментов представлены в виде таблицы:}

\begin{table}[h]
\centering
\begin{tabular}{|c|c|c|c|}
\hline
\multirow{2}{*}{Размер данных} & \multicolumn{3}{c|}{Время выполнения (секунды)} \\ \cline{2-4} 
                               & OpenMP & TBB   & STL  \\ \hline
1000                           & 0.015  & 0.017 & 0.018 \\ \hline
10000                          & 0.098  & 0.102 & 0.105 \\ \hline
100000                         & 1.212  & 1.237 & 1.254 \\ \hline
1000000                        & 14.127 & 14.203& 14.314\\ \hline
\end{tabular}
\caption{Сравнение времени выполнения различных версий алгоритма}
\label{table:results}
\end{table}

\newpage

% Анализ данных
\section*{Анализ данных}
\addcontentsline{toc}{section}{Анализ данных}
\par Проведенные эксперименты показали, что параллельные версии алгоритма построения выпуклой оболочки методом прохода Грэхема значительно ускоряют процесс вычислений по сравнению с последовательной реализацией. В частности, для больших наборов данных выигрыш в производительности особенно заметен. Среди рассмотренных технологий наилучшие результаты показала версия, реализованная с использованием OpenMP, что объясняется эффективностью распараллеливания операций сортировки и обработки точек. Тем не менее, все три параллельные реализации (OpenMP, TBB и STL) продемонстрировали схожие результаты, что подтверждает высокую эффективность применяемых методов распараллеливания.

\newpage

% Заключение
\section*{Заключение}
\addcontentsline{toc}{section}{Заключение}
\par В ходе выполнения данной работы была реализована и проанализирована эффективность различных параллельных версий алгоритма построения выпуклой оболочки методом прохода Грэхема. Экспериментальные данные показали, что использование параллельных технологий (OpenMP, TBB и STL) позволяет значительно ускорить вычисления по сравнению с последовательной реализацией. Наилучшие результаты были достигнуты при использовании OpenMP. Результаты данной работы могут быть использованы для дальнейшего исследования и оптимизации алгоритмов вычислительной геометрии в условиях параллельных вычислений.

\newpage

% Литература
\addcontentsline{toc}{section}{Литература}
\begin{thebibliography}{9}
\bibitem{reference1} 
Постников М.М., Введение в вычислительную геометрию. — М.: МЦНМО, 2012.
\bibitem{reference2} 
Браун К., Алгоритмы: построение и анализ. — М.: МЦНМО, 2011.
\bibitem{reference3} 
Томас Х. Кормен, Чарльз И. Лейзерсон, Рональд Л. Ривест, Клифорд Штайн, Алгоритмы: построение и анализ. — М.: Вильямс, 2005.
\bibitem{reference4} 
OpenMP API specification for parallel programming. Available: https://www.openmp.org/
\bibitem{reference5} 
Intel Threading Building Blocks (Intel TBB). Available: https://www.threadingbuildingblocks.org/
\end{thebibliography}


\newpage

\section*{Приложение}
\addcontentsline{toc}{section}{Приложение}

\begin{lstlisting}[language=C++,caption=Последовательная версия]
// Copyright 2024 Kistrimova Ekaterina
#include "seq/kistrimova_e_graham_alg/include/ops_seq.hpp"

bool GrahamAlgTask::pre_processing() {
  internal_order_test();
  input = std::vector<point>(taskData->inputs_count[0]);
  auto* in = reinterpret_cast<point*>(taskData->inputs[0]);
  for (unsigned int i = 0; i < taskData->inputs_count[0]; i++) {
    input[i] = in[i];
  }
  output = input;
  return true;
}

bool GrahamAlgTask::validation() {
  internal_order_test();
  return taskData->outputs_count[0] <= taskData->inputs_count[0];
}

bool GrahamAlgTask::run() {
  internal_order_test();
  output = graham(input);
  return true;
}

bool GrahamAlgTask::post_processing() {
  internal_order_test();
  std::copy(output.begin(), output.end(), reinterpret_cast<point*>(taskData->outputs[0]));
  return true;
}

double rotate(point X, point Y, point Z) { return (Y.x - X.x) * (Z.y - Y.y) - (Y.y - X.y) * (Z.x - Y.x); }

std::vector<point> graham(std::vector<point> points) {
  int n = points.size();
  std::vector<int> R(n);
  for (int i = 0; i < n; i++) R[i] = i;

  for (int i = 1; i < n; i++) {
    if (points[R[i]].x < points[R[0]].x) std::swap(R[i], R[0]);
  }

  for (int i = 2; i < n; i++) {
    int j = i;
    while (j > 1 && rotate(points[R[0]], points[R[j - 1]], points[R[j]]) < 0) {
      std::swap(R[j], R[j - 1]);
      j--;
    }
  }

  std::vector<point> res{points[R[0]], points[R[1]]};
  for (int i = 2; i < n; i++) {
    while (rotate(res.end()[-2], res.end()[-1], points[R[i]]) <= 0) res.pop_back();
    res.push_back(points[R[i]]);
  }

  return res;
}
\end{lstlisting}

\newpage

\begin{lstlisting}[language=C++,caption=OpenMP версия]
// Copyright 2024 Kistrimova Ekaterina
#include <omp.h>

#include "omp/kistrimova_e_graham_alg_omp/include/ops_omp.hpp"

bool GrahamAlgTask::pre_processing() {
  internal_order_test();
  input = std::vector<point>(taskData->inputs_count[0]);
  auto* in = reinterpret_cast<point*>(taskData->inputs[0]);
  for (unsigned int i = 0; i < taskData->inputs_count[0]; i++) {
    input[i] = in[i];
  }
  output = input;
  return true;
}

bool GrahamAlgTask::validation() {
  internal_order_test();
  return taskData->outputs_count[0] <= taskData->inputs_count[0];
}

bool GrahamAlgTask::run() {
  internal_order_test();
  output = graham(input);
  return true;
}

bool GrahamAlgTask::post_processing() {
  internal_order_test();
  std::copy(output.begin(), output.end(), reinterpret_cast<point*>(taskData->outputs[0]));
  return true;
}

double rotate(point X, point Y, point Z) { return (Y.x - X.x) * (Z.y - Y.y) - (Y.y - X.y) * (Z.x - Y.x); }

std::vector<point> graham(std::vector<point> points) {
  int n = points.size();
  std::vector<int> R(n);
#pragma omp parallel for
  for (int i = 0; i < n; i++) R[i] = i;

#pragma omp parallel for
  for (int i = 1; i < n; i++) {
    if (points[R[i]].x < points[R[0]].x) {
#pragma omp critical
      { std::swap(R[i], R[0]); }
    }
  }

  for (int i = 2; i < n; i++) {
    int j = i;
    while (j > 1 && rotate(points[R[0]], points[R[j - 1]], points[R[j]]) < 0) {
      std::swap(R[j], R[j - 1]);
      j--;
    }
  }

  std::vector<point> res{points[R[0]], points[R[1]]};
  for (int i = 2; i < n; i++) {
    while (rotate(res.end()[-2], res.end()[-1], points[R[i]]) < 0 && !res.empty()) res.pop_back();
    res.push_back(points[R[i]]);
  }
  return res;
}
\end{lstlisting}

\newpage

\begin{lstlisting}[language=C++,caption=TBB версия]
// Copyright 2024 Kistrimova Ekaterina
#include <tbb/tbb.h>

#include "tbb/kistrimova_e_graham_alg_tbb/include/ops_tbb.hpp"

bool GrahamAlgTask::pre_processing() {
  internal_order_test();
  input = std::vector<point>(taskData->inputs_count[0]);
  auto* in = reinterpret_cast<point*>(taskData->inputs[0]);
  for (unsigned int i = 0; i < taskData->inputs_count[0]; i++) {
    input[i] = in[i];
  }
  output = input;
  return true;
}

bool GrahamAlgTask::validation() {
  internal_order_test();
  return taskData->outputs_count[0] <= taskData->inputs_count[0];
}

bool GrahamAlgTask::run() {
  internal_order_test();
  output = graham(input);
  return true;
}

bool GrahamAlgTask::post_processing() {
  internal_order_test();
  std::copy(output.begin(), output.end(), reinterpret_cast<point*>(taskData->outputs[0]));
  return true;
}

double rotate(point X, point Y, point Z) { return (Y.x - X.x) * (Z.y - Y.y) - (Y.y - X.y) * (Z.x - Y.x); }

std::vector<point> graham(std::vector<point> points) {
  int n = points.size();
  std::vector<int> R(n);
  tbb::parallel_for(0, n, 1, [&](int i) { R[i] = i; });

  int min_x_idx = tbb::parallel_reduce(
      tbb::blocked_range<int>(1, n), 0,
      [&points, &R](const tbb::blocked_range<int>& range, int init) -> int {
        for (int i = range.begin(); i != range.end(); ++i) {
          if (points[R[i]].x < points[R[init]].x) {
            init = i;
          }
        }
        return init;
      },
      [&points, &R](int idx1, int idx2) -> int {
        return (points[R[idx1]].x < points[R[idx2]].x ||
                (points[R[idx1]].x == points[R[idx2]].x && points[R[idx1]].y < points[R[idx2]].y))
                   ? idx1
                   : idx2;
      });
  std::swap(R[0], R[min_x_idx]);

  tbb::parallel_sort(R.begin() + 1, R.end(),
                     [&](int a, int b) { return rotate(points[R[0]], points[a], points[b]) > 0; });

  std::vector<point> res{points[R[0]], points[R[1]]};
  for (int i = 2; i < n; i++) {
    while (rotate(res.end()[-2], res.end()[-1], points[R[i]]) <= 0) {
      res.pop_back();
    }
    res.push_back(points[R[i]]);
  }

  return res;
}
\end{lstlisting}

\newpage

\begin{lstlisting}[language=C++,caption=STL версия]
// Copyright 2024 Kistrimova Ekaterina
#include <algorithm>
#include <cmath>
#include <mutex>
#include <thread>

#include "stl/kistrimova_e_graham_alg_stl/include/ops_stl.hpp"

bool GrahamAlgTask::pre_processing() {
  internal_order_test();
  input = std::vector<point>(taskData->inputs_count[0]);
  auto* in = reinterpret_cast<point*>(taskData->inputs[0]);
  for (unsigned int i = 0; i < taskData->inputs_count[0]; i++) {
    input[i] = in[i];
  }
  output = input;
  return true;
}

bool GrahamAlgTask::validation() {
  internal_order_test();
  return taskData->outputs_count[0] <= taskData->inputs_count[0];
}

bool GrahamAlgTask::run() {
  internal_order_test();
  output = graham(input);
  return true;
}

bool GrahamAlgTask::post_processing() {
  internal_order_test();
  std::copy(output.begin(), output.end(), reinterpret_cast<point*>(taskData->outputs[0]));
  return true;
}

double rotate(point X, point Y, point Z) { return (Y.x - X.x) * (Z.y - Y.y) - (Y.y - X.y) * (Z.x - Y.x); }

void compute_angles(std::vector<point>& points, std::vector<int>& R, int start, int end, const point& p0,
                    std::vector<double>& angles) {
  for (int i = start; i < end; ++i) {
    double dx = points[R[i]].x - p0.x;
    double dy = points[R[i]].y - p0.y;
    angles[i] = std::atan2(dy, dx);
  }
}

std::vector<point> graham(std::vector<point> points) {
  int n = points.size();
  std::vector<int> R(n);
  for (int i = 0; i < n; i++) R[i] = i;

  auto p0_iter = std::min_element(points.begin(), points.end(), [](const point& a, const point& b) {
    return a.x < b.x || (a.x == b.x && a.y < b.y);
  });
  std::swap(points[0], *p0_iter);
  std::swap(R[0], R[std::distance(points.begin(), p0_iter)]);

  std::vector<double> angles(n);
  int num_threads = std::thread::hardware_concurrency();
  std::vector<std::thread> threads(num_threads);
  int chunk_size = n / num_threads;

  for (int i = 0; i < num_threads; ++i) {
    int start = i * chunk_size;
    int end = (i == num_threads - 1) ? n : (i + 1) * chunk_size;
    threads[i] = std::thread(compute_angles, std::ref(points), std::ref(R), start, end, points[0], std::ref(angles));
  }

  for (auto& t : threads) {
    t.join();
  }

  std::sort(R.begin() + 1, R.end(), [&angles](int i, int j) { return angles[i] < angles[j]; });
  std::vector<point> res{points[R[0]], points[R[1]]};
  for (int i = 2; i < n; i++) {
    while (rotate(res.end()[-2], res.end()[-1], points[R[i]]) <= 0) {
      res.pop_back();
    }
    res.push_back(points[R[i]]);
  }

  return res;
}
\end{lstlisting}

\end{document}