% !TeX program = pdflatex
% !TeX TXS-program:compile = txs:///pdflatex/[-shell-escape]
\documentclass[12pt]{article}
\usepackage{minted}
\usepackage{url}
\usepackage{adjustbox}
\usepackage{amsfonts}
\usepackage{listings}
\usepackage{caption}
\usepackage{multirow}
\setminted[python]{breaklines}
\usepackage{hyperref}
\usepackage[utf8]{inputenc}
\usepackage[russian]{babel}
\usepackage[left=25mm, top=20mm, right=15mm, bottom=20mm, nohead, footskip=10mm]{geometry}
\usepackage{indentfirst}
\linespread{1.3}
\setlength{\parindent}{1.25cm}
\newenvironment{longlisting}{\captionsetup{type=listing, name=Приложение}}{}
\DeclareCaptionLabelSeparator{def}{. }
\captionsetup[figure]{name=Рис., labelsep=def}
\captionsetup[listing]{name=Приложение}
\captionsetup[table]{name=Таблица, labelsep=def}
\begin{document}
\begin{center}
    \large
    МИНИСТЕРСТВО НАУКИ И ВЫСШЕГО ОБРАЗОВАНИЯ РОССИЙСКОЙ ФЕДЕРАЦИИ \\
    Федеральное государственное автономное образовательное учреждение высшего образования \\
    \textbf{«Нижегородский государственный университет им. Н.И. Лобачевского»}

    \vspace{0.5cm}

    \textbf{Институт информационных технологий, математики и механики}
    
    \vspace{0.5cm}
    
    Направление подготовки: <<01.03.02 Прикладная математика и информатика>>

    \vspace{1.5cm}

    \textbf{ОТЧЁТ}

    по лабораторной работе

    \vspace{0.5cm}

    \textbf{на тему:}

    \textbf{<<Параллельная реализация фильтра Гаусса (ядро 3 $\times$ 3, блочное разбиение).>>}
\end{center}

\vspace{2.5cm}

\begin{flushright}
    Выполнил: \\
    студент группы 3821Б1ПМоп3 \\
    Мухин Иван Сергеевич

    \vspace{0.5cm}

    Проверил: \\
    аспирант \\
    Нестеров Александр Юрьевич
    
    \vspace{0.5cm}

\end{flushright}

\vfill

\begin{center}
    Нижний Новгород \\
    2024
\end{center}
\thispagestyle{empty}

\newpage
     
    \tableofcontents % Вывод содержания

\newpage
\begin{center}
\section*{Введение}
\end{center}
\addcontentsline{toc}{section}{Введение}
\par
В настоящее время различные алгоритмы обработки изображений находят широкое применение в современном мире. Однако с постепенным прогрессом изображения становятся всё более качественными и, как следствие, более <<весомые>>, что определенно влияет на скорость применения различных последовательных алгоритмов к ним, но вместе с течением прогресса улучшаются характеристики процессоров. Для оптимизации алгоритма его можно распараллелить, то есть распределить участки счета между вычислительными узлами.
\par
\textit{\textbf{Целью настоящей работы}} является рассмотрение различных библиотек для распараллеливания вычислений на примере фильтра Гаусса с ядром $3 \times 3$ и блочным разбиением изображения. Тогда можно поставить задачи:
\begin{enumerate}
	\item Сформулировать постановку задачи обработки изображения
	\item Реализовать последовательную версию алгоритма.
	\item Реализовать параллельную версию средствами OpenMP.
	\item Реализовать параллельную версию средствами TBB.
	\item Реализовать параллельную версию средствами STL.
\end{enumerate}

\newpage
\section{Постановка задачи}
\par
Размытие по Гауссу в цифровой обработке изображений — способ размытия изображения с помощью функции Гаусса, названной в честь немецкого математика Карла Фридриха Гаусса.
\par
Этот эффект широко используется в графических редакторах для уменьшения шума изображения и снижения детализации. Визуальный эффект этого способа размытия напоминает эффект просмотра изображения через полупрозрачный экран, и отчётливо отличается от эффекта боке, создаваемого расфокусированным объективом или тенью объекта при обычном освещении.
\par
Размытие по Гауссу также используется в качестве этапа предварительной обработки в алгоритмах компьютерного зрения для улучшения структуры изображения в различных масштабах.
Применение размытия по Гауссу к изображению математически аналогично свёртке изображения с помощью функции Гаусса. Оно также известно как двумерное преобразование Вейерштрасса. Циклическая свёртка (т. е. круговое размытие по рамке), напротив, более точно воспроизводит эффект боке.

Поскольку преобразование Фурье функции Гаусса само является функцией Гаусса, применение размытия по Гауссу приводит к уменьшению высокочастотных компонентов изображения. Таким образом, размытие по Гауссу является фильтром нижних частот.


\newpage
\section{Программная реализация}
Для хранения изображения создадим два вспомогательных класса: <<Pixel>>\hyperref[lst:pixel]{(Приложение 1)} и <<PixelMap>>\hyperref[lst:pixel_map]{(Приложение 2)}. В классе <<Pixel>> хранятся значения <<uint8\_t>>, принимающие значения от 0 до 255, которые отвечают за интенсивность цветов в формате RGB (красный, зеленый, синий). Класс <<PixelMap>> хранит ширину и высоту изображения, а их значения соответствуют количеству пикселей.
\par
Для расчета ядра фильтра используется функция <<create\_gaussian\_kernel>>\hyperref[lst:kernel]{(Приложение 3)}, которая непосредственно и размывает изображение путем применения экспоненциального преобразования, функция которого соответствует плотности нормального распределения. Далее используется функция для создания новых пикселей для изображения\hyperref[lst:apply]{(Приложение 4)}, которая применяет матричное ядро к матрице из пикселей идентичного размера.
\par
До текущего момента все вспомогательные классы являются общими для всех версий, меняться будет лишь применение фильтра к изображению. Для параллельной версии будет использоваться алгоритм блочного разбиения, сутью которого является разбиение изображения на блоки, по которым будут работать потоки и не пересекаться друг с другом.


\subsection{Последовательная версия}
\par
В последовательной версии всё просто: применяем фильтр Гаусса с ядром $3 \times 3$ к каждому пикселю последовательно в цикле\hyperref[lst:seq]{(Приложение 5)}.


\subsection{OpenMP версия}
\textit{OpenMP}\cite{omp} -- открытый стандарт для распараллеливания программ на языках С, С++ и Фортран. Даёт описание совокупности директив компилятора, библиотечных процедур и переменных окружения, которые предназначены для программирования многопоточных приложений на многопроцессорных системах с общей памятью. OpenMP реализует параллельные вычисления с помощью многопоточности, в которой ведущий (англ. master) поток создаёт набор ведомых потоков, и задача распределяется между ними. Предполагается, что потоки выполняются параллельно на машине с несколькими процессорами (количество процессоров не обязательно должно быть больше или равно количеству потоков).
\par
Для распаралеллеливания будем использовать 4 потока. Алгоритм следующий:
\begin{enumerate}
	\item Разбиваем изображение на 4 независимых блока.
	\item Создаем новый пиксель путем применения фильтра Гаусса. 
\end{enumerate}
\par
Каждому блоку поставим в соответствие номер потока, поэтому переменная <<ThreadID>> будет для каждого потока своя\hyperref[lst:omp]{(Приложение 6)}. Соответственно, для каждого блока можно определить специфичную индексацию с помощью переменных <<i\_start>> и <<j\_start>>. Директива <<pragma omp parallel for private(ThreadID)>> является ключевой и распределяет работу между потоками.

\subsection{TBB версия}
\textit{Intel Threading Building Blocks (также известная как TBB)}\cite{tbb} — кроссплатформенная библиотека шаблонов C++, разработанная компанией Intel для параллельного программирования. Библиотека содержит алгоритмы и структуры данных, позволяющие программисту избежать многих сложностей, возникающих при использовании традиционных реализаций потоков, таких как POSIX Threads, Windows threads или Boost Threads, в которых создаются отдельные потоки исполнения, синхронизируемые и останавливаемые вручную. Библиотека TBB абстрагирует доступ к отдельным потокам.
\par
Алгоритм идентичен OpenMP, однако у TBB есть особенность, так называемое \textit{итерационное пространство}, с помощью него и будет определяться специфичный итератор, который будет распределять работу между потоками. Для начала нужно создать экземпляр класса <<task\_arena>>\hyperref[lst:tbb]{(Приложение 7)}, после чего в его метод <<execute>> передать итерационное пространство и лямбда-функцию, внутри которой будет применятся фильтр по схеме, которая аналогична OpenMP.


\subsection{STL версия}
Распараллелить также можно, используя C++ std::threads. Она включена в состав STL (Standard Template Library). Схема полностью аналогична описанным выше библиотекам, отличие состоит в том, что создается массив указателей, каждый из которых представляет из себя программный интерфейс для одного потока. После завершения вычислений производится <<join>> и массив потоков удаляется\hyperref[lst:stl]{(Приложение 8)}.
\newpage
\begin{center}
\section*{Заключение}
\end{center}
\addcontentsline{toc}{section}{Заключение}
Таким образом, в рамках данной работы были рассмотрены различные библиотеки для распараллеливания вычислений, которые были использованы для оптимизации алгоритма блочного разбиения фильтра Гаусса с ядром $3 \times 3$.

\newpage
\addcontentsline{toc}{section}{Список литературы}

\begin{thebibliography}{9}
	\bibitem{omp} OpenMP -- \url{https://www.openmp.org/}
	\bibitem{tbb} Intel Threading Building Blocks -- \url{https://github.com/oneapi-src/oneTBB}
\end{thebibliography}


\newpage
\section*{Приложения}
\addcontentsline{toc}{section}{Приложения}

\begin{longlisting}
\begin{minted}[frame=lines,framesep=2mm,
baselinestretch=1.2,
fontsize=\footnotesize,
linenos]{cpp}
class Pixel {
 public:
  uint8_t r, g, b;

  friend bool operator==(const Pixel& a, const Pixel& b);

  friend bool operator!=(const Pixel& a, const Pixel& b) { return !(a == b); }
};
\end{minted}
\caption{Программный интерфейс для пикселя}
\label{lst:pixel}
\end{longlisting}

\begin{longlisting}
\begin{minted}[frame=lines,framesep=2mm,
baselinestretch=1.2,
fontsize=\footnotesize,
linenos]{cpp}
class PixelMap {
 private:
  uint32_t n;
  uint32_t m;

  void set_data(uint8_t r, uint8_t g, uint8_t b);

 public:
  std::vector<Pixel> data;
  PixelMap() : n(0), m(0){};

  PixelMap(uint32_t n_, uint32_t m_, uint8_t r = 0, uint8_t g = 0, uint8_t b = 0);
  uint32_t width() const { return n; }
  uint32_t height() const { return m; }
  ~PixelMap() = default;

  Pixel& get_pixel(uint32_t i, uint32_t j);

  const Pixel& get_pixel(uint32_t i, uint32_t j) const;

  friend bool operator==(const PixelMap& a, const PixelMap& b);

  friend bool operator!=(const PixelMap& a, const PixelMap& b) { return !(a == b); }
};
\end{minted}
\caption{Программный интерфейс для изображения}
\label{lst:pixel_map}
\end{longlisting}

\begin{longlisting}
\begin{minted}[frame=lines,framesep=2mm,
baselinestretch=1.2,
fontsize=\footnotesize,
linenos]{cpp}
void create_gaussian_kernel() {
  double sigm = 2.0;
  double norm = 0.0;
  for (int i = -rad; i <= rad; i++) {
    for (int j = -rad; j <= rad; j++) {
      kernel[i + rad][j + rad] = (std::exp(-(i * i + j * j) / (2 * sigm * sigm)));
      norm += kernel[i + rad][j + rad];
    }
  }
  for (uint32_t i = 0; i < kern_size; i++) {
    for (uint32_t j = 0; j < kern_size; j++) {
      kernel[i][j] /= norm;
    }
  }
}
\end{minted}
\caption{Функция для создания ядра фильтра Гаусса}
\label{lst:kernel}
\end{longlisting}


\begin{longlisting}
\begin{minted}[frame=lines,framesep=2mm,
baselinestretch=1.2,
fontsize=\footnotesize,
linenos]{cpp}
Pixel get_new_pixel(uint32_t w, uint32_t h) {
  double result_r = 0;
  double result_b = 0;
  double result_g = 0;
  for (int i = -rad; i <= rad; i++) {
    for (int j = -rad; j <= rad; j++) {
      uint32_t new_h = h + j;
      uint32_t new_w = w + i;
      new_h = clamp(new_h, width_input);
      Pixel neighborColor = image.get_pixel(new_w, new_h);
      result_r += neighborColor.r * kernel[i + rad][j + rad];
      result_b += neighborColor.b * kernel[i + rad][j + rad];
      result_g += neighborColor.g * kernel[i + rad][j + rad];
    }
  }
  return Pixel({(uint8_t)std::round(result_r),(uint8_t)std::round(result_g),
                (uint8_t)std::round(result_b)});
}
\end{minted}
\caption{Функция для применения фильтра Гаусса}
\label{lst:apply}
\end{longlisting}


\begin{longlisting}
\begin{minted}[frame=lines,framesep=2mm,
baselinestretch=1.2,
fontsize=\footnotesize,
linenos]{cpp}
void filter_to_image() {
  for (uint32_t i = 0; i < width_input; i++) {
    for (uint32_t j = 0; j < height_input; j++) {
      image.get_pixel(i, j) = get_new_pixel(i, j);
    }
  }
}
\end{minted}
\caption{Последовательная версия применения фильтра}
\label{lst:seq}
\end{longlisting}


\begin{longlisting}
\begin{minted}[frame=lines,framesep=2mm,
baselinestretch=1.2,
fontsize=\footnotesize,
linenos]{cpp}
void filter_to_image() {
  int Size = static_cast<int>(width_input);
  int GridThreadsNum = 4;
  int ThreadID;
  int GridSize = static_cast<int>(std::sqrt(static_cast<double>(GridThreadsNum)));
  int BlockSize = Size / GridSize;
  omp_set_num_threads(GridThreadsNum);
#pragma omp parallel private(ThreadID)
  {
    ThreadID = omp_get_thread_num();
    int i_start = static_cast<uint32_t>((ThreadID / GridSize) * BlockSize);
    int j_start = static_cast<uint32_t>((ThreadID % GridSize) * BlockSize);
    for (int i = 0; i < BlockSize; i++) {
      for (int j = 0; j < BlockSize; j++) {
        auto ii = static_cast<uint32_t>(i);
        auto jj = static_cast<uint32_t>(j);
        image.get_pixel(ii + i_start, jj + j_start) = get_new_pixel(ii + i_start, jj + j_start);
      }
    }
  }
}
\end{minted}
\caption{OpenMP версия применения фильтра}
\label{lst:omp}
\end{longlisting}


\begin{longlisting}
\begin{minted}[frame=lines,framesep=2mm,
baselinestretch=1.2,
fontsize=\footnotesize,
linenos]{cpp}
void filter_to_image() {
  int GridSize = static_cast<int>(std::sqrt(static_cast<double>(4)));
  int BlockSize = width_input / GridSize;
  tbb::task_arena arena(4);
  arena.execute([&] {
    tbb::parallel_for(tbb::blocked_range<uint32_t>(0, 3), [&](const tbb::blocked_range<uint32_t>& r) {
      for (uint32_t k = r.begin(); k < r.end(); k++) {
        auto thread_id = tbb::this_task_arena::current_thread_index();
        auto i_start = static_cast<uint32_t>((thread_id / GridSize) * BlockSize);
        auto j_start = static_cast<uint32_t>((thread_id % GridSize) * BlockSize);
        for (int i = 0; i < BlockSize; i++) {
          for (int j = 0; j < BlockSize; j++) {
            image.get_pixel(i + i_start, j + j_start) = get_new_pixel(i + i_start, j + j_start);
          }
        }
      }
    });
  });
}
\end{minted}
\caption{TBB версия применения фильтра}
\label{lst:tbb}
\end{longlisting}



\begin{longlisting}
\begin{minted}[frame=lines,framesep=2mm,
baselinestretch=1.2,
fontsize=\footnotesize,
linenos]{cpp}
void filter_to_image() {
  int num_threads = 4;
  auto* threads = new std::thread[num_threads];
  int BlockSize = width_input / 2;
  for (int k = 0; k < num_threads; k++) {
    auto i_start = static_cast<uint32_t>((k / 2) * BlockSize);
    auto j_start = static_cast<uint32_t>((k % 2) * BlockSize);
    threads[k] = std::thread([this, BlockSize, i_start, j_start]() {
      for (uint32_t i = 0; i < static_cast<uint32_t>(BlockSize); i++) {
        for (uint32_t j = 0; j < static_cast<uint32_t>(BlockSize); j++) {
          image.get_pixel(i + i_start, j + j_start) = get_new_pixel(i + i_start, j + j_start);
        }
      }
    });
  }
  for (int i = 0; i < num_threads; i++) {
    threads[i].join();
  }
  delete[] threads;
}
\end{minted}
\caption{STL версия применения фильтра}
\label{lst:stl}
\end{longlisting}








\end{document}