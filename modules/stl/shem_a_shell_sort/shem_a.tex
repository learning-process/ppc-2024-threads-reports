\documentclass[]{article}

\usepackage[warn]{mathtext}
\usepackage[T2A]{fontenc}
\usepackage[utf8]{luainputenc}
\usepackage[english, russian]{babel}
\usepackage[pdfpagemode=UseNone,colorlinks,allcolors=black]{hyperref}
\usepackage{breakurl}
\usepackage[12pt]{extsizes}
\usepackage{color}
\usepackage{geometry}
\usepackage{enumitem}
\usepackage{multirow}
\usepackage{graphicx}
\usepackage{indentfirst}
\usepackage{amsmath}
\usepackage{amssymb}
\usepackage{listings}

\geometry{a4paper,top=2cm,bottom=2cm,left=3cm,right=1.5cm}
\setlength{\parskip}{0.5cm}
\setlist{nolistsep, itemsep=0.3cm,parsep=0pt}

\makeatletter
\renewcommand\@biblabel[1]{#1.\hfil}
\makeatother

\usepackage{amsthm}

\theoremstyle{remark}
\newtheorem*{remark}{Примечание}

\theoremstyle{definition}
\newtheorem*{defi}{Определение}

\usepackage{titlesec}
\titlelabel{\thetitle.\quad}

\newcommand{\term}[1]{$\mathtt{#1}$}
\newcommand{\termn}[1]{\mathtt{#1}}

\newcommand{\cpp}{\textit{}}

\begin{document}

\begin{titlepage}

\begin{center}
\MakeUppercase{Министерство науки и высшего образования Российской~Федерации}
\end{center}

\begin{center}
Федеральное государственное автономное образовательное учреждение высшего~образования \\
\textbf{Национальный исследовательский Нижегородский государственный университет им.~Н.И. Лобачевского}
\end{center}

\begin{center}
\textbf{Институт информационных технологий, математики и механики}
\end{center}
\begin{center}
\textbf{Кафедра математического обеспечения и суперкомпьютерных технологий}
\end{center}
\begin{center}
Направление подготовки: <<Прикладная математика и информатика>>
\end{center}

\vspace{3em}

\begin{center}
\textbf{\Large\MakeUppercase{Отчёт по лабораторной работе}} \\
\end{center}
\begin{center}
\textbf{\Large<<Сортировка Шелла с простым слиянием>>} \\
\end{center}

\vspace{5em}

\newbox{\lbox}
\savebox{\lbox}{\hbox{text}}
\newlength{\maxl}
\setlength{\maxl}{\wd\lbox}
\hfill\parbox{7cm}{
\hspace*{5cm}\hspace*{-5cm}\textbf{Выполнил:} \\ студентка группы 	3821Б1ПМоп3 \\ Шемякина А. П.\\
\\

\hspace*{5cm}\hspace*{-5cm}\textbf{Проверил:} \\ Нестеров А. Ю.
}
\vspace{\fill}

\begin{center} Нижний Новгород \\ 2024 \end{center}

\end{titlepage}

\setcounter{page}{2}

% Содержание
\tableofcontents
\newpage

\section{Введение}

\par Алгоритмы сортировки — это алгоритмы, которые упорядочивают данные в определенном порядке. Они используются во многих задачах, таких как: сортировка элементов в списке и строк в текстовом файле, сортировка чисел в массиве и результатов поиска. Они позволяют оптимизировать процессы работы с данными, сокращая временные затраты и упрощая структурирование информации и являются неотъемлемой частью таких базовых операций, как поиск, фильтрация, агрегация и анализ данных.

\par Сортировка Шелла - это модифицированная сортировка простыми вставками. Она включает в себя сортировку вставкой подгрупп элементов, но только в подгруппе они идут не в ряд, а равномерно выбираются с некоторой дельтой по индексу. После первоначальных грубых проходов, дельта методично уменьшается, пока расстояние между элементами этих несвязных подмножеств не достигнет единицы. Благодаря первоначальным проходам с большим шагом, большинство малых по значению элементов перебрасываются в левую часть массива, большинство крупных элементов массива попадают в правую.

\par Постановка задачи:
\par В данной лабораторной работе необходимо рассмотреть и реализовать алгоритм сортировки Шелла в трёх вариантах:

\par 1) Последовательная версия — классическая реализация алгоритма без использования параллелизма;
\par 2) Параллельная версия на OpenMP — версия алгоритма с использованием директив OpenMP и простым слиянием для распараллеливания вычислений;
\par 3) Параллельная версия на TBB — реализация алгоритма с использованием библиотеки Threading Building Blocks и простым слиянием для распараллеливания вычислений.

\par В рамках данной работы необходимо выполнить следующие задачи:
\begin{enumerate}
    \item Реализовать, исследовать и сравнить производительность трёх реализаций алгоритма сортировки Шелла с использованием языка программирования C++;
    \item Добавить параллельную реализацию метода на языке программирования C++ с использованием разных технологий написания параллельных программ для с общей памятью: OpenMP, TBB;
    \item Сравнить результаты и время работы разных реализаций алгоритма на различных объёмах данных, чтобы сделать выводы об эффективности параллелизации алгоритма Шелла при использовании OpenMP и TBB.
\end{enumerate}

\newpage
\section{Алгоритмы, используемые для решения поставленной задачи}
\subsection{Анализ алгоритма сортировки Шелла}

\par 1) Временная сложность. Средняя и худшая временная сложность сортировки Шелла зависит от выбора последовательности шагов. В классическом варианте, когда шаги выбираются как \term{\frac{n}{2^k}}, средняя сложность составляет \term{O(n \log^2 n)}. Однако с оптимальными последовательностями шагов, такими как последовательность Хиббарда, можно достичь времени \term{O(n^{3/2})} или даже лучше.
\par 2) Пространственная сложность. Алгоритм требует \term{O(1)} дополнительной памяти, так как является сортировкой на месте.
\par 3) Устойчивость. Сортировка Шелла не является устойчивой, так как элементы, равные по значению, могут меняться местами в процессе сортировки.
\par 4) Адаптивность. Сортировка Шелла хорошо адаптируется к частично отсортированным массивам, быстро уменьшая количество инверсий.

\par Сортировка Шелла подходит для средних по размеру массивов и в ситуациях, когда доступ к элементам массива происходит не последовательно, а с пропусками, что делает её особенно полезной в некоторых специфических задачах.

\subsection{Простое слияние}

\par Простое слияние  — это базовый алгоритм слияния двух отсортированных массивов в один отсортированный массив. Этот процесс является фундаментальной операцией в многих алгоритмах сортировки, наиболее известным из которых является сортировка слиянием Главное преимущество сортировки слиянием — она работает всегда с одной и той же скоростью на любых массивах.

\par Описание алгоритма простого слияния:

\par Пусть есть два отсортированных массива \term{A} и \term{B}. С помощью простого слияния объединим эти массивы в один отсортированный массив \term{C}. Процесс простого слияния включает следующие шаги:

\begin{enumerate}
    \item Инициализация указателей. Создаются два указателя \term{i} и \term{j}, которые изначально указывают на первые элементы массивов \term{A} и \term{B} соответственно. Также создается указатель \term{k}, указывающий на текущую позицию в результирующем массиве \term{C};
    \item Сравнение элементов. Сравниваются элементы \term{A[i]} и \term{A[j]}. Меньший из двух элементов копируется в \term{C[k]}, после чего указатель на этот элемент и указатель \term{k} инкрементируются (увеличиваются на 1). Если \term{A[i] \leq B[j]}, то \term{C[k] = A[i]} и \term{i} инкрементируется; иначе \term{C[k] = B[j]} и \term{j} инкрементируется;
    \item Обработка оставшихся элементов. Когда один из массивов полностью скопирован в \term{C}, оставшиеся элементы другого массива копируются в \term{C}. Если \term{i} еще не достигло конца \term{A}, оставшиеся элементы \term{A[i], A[i+1], \ldots} копируются в \term{C}. Аналогично, если \term{j} еще не достигло конца \term{B}, оставшиеся элементы \term{B[j], B[j+1], \ldots} копируются;
    \item Результат. В конце алгоритма \term{C} содержит все элементы из \term{A} и \term{B}, упорядоченные по возрастанию.
\end{enumerate}

\subsection{Распараллеливание}

\par Принцип распараллеливания в данной работе заключается в разделении исходного массива на несколько частей, сортировке каждой части в отдельном потоке и последующем объединении отсортированных частей:
\par 1) Массив разбивается на части равного размера;
\par 2) Последняя часть получает дополнительные элементы из остатка;
\par 3) Каждая часть массива сортируется в отдельном потоке;
\par 4) После завершения параллельной сортировки каждой части исходного массива, они объединяются в один отсортированный массив с помощью простого слияния.

\newpage

\section{Программная реализация основных алгоритмов}

\subsection{Последовательная реализация}

\par Функция последовательной реализации имеет вид: \cpp{void ShellTaskOMP::ShellSort(std::vector<int>\& vec)}. Входные данные в функцию: \cpp{std::vector<int>\& vec} -- вектор целых чисел, который нужно отсортировать.
\par Сначала первый интервал (\cpp{interval}) устанавливается как половина размера вектора \cpp{vec}. Далее он постепенно уменьшается вдвое на каждой итерации внешнего цикла, пока не станет равным нулю.
\par Для каждого значения \cpp{interval} выполняется внутренняя сортировка вставками, которая сравнивает элементы с шагом \cpp{interval}, а внутренний цикл сортировки вставками перемещает элементы на позиции, упорядоченные с учетом текущего интервала.


\subsection{Реализация простого слияния}

\begin{lstlisting}[language=C++]
std::vector<int> ShellTaskOMP::merge(const std::vector<std::vector<int>>& chunks):
\end{lstlisting}

\par Входные данные: \cpp{const std::vector<std::vector<int>>\& chunks} -- вектор, содержащий несколько отсортированных векторов, которые нужно будет объединить.

\par Выходные данные: \cpp{std::vector<int> res} -- отсортированный массив, полученный путем объединения всех векторов.


\subsection{Реализация OpenMP}

\par Параллельная сортировка векторов реализуется с использованием директивы OpenMP \cpp{\#pragma omp parallel for} для параллельного выполнения цикла. Каждый поток сортирует свой вектор с помощью последовательной сортировки Шелла.
\par Синхронизация реализуется с помощью барьер синхронизации \cpp{\#pragma omp barrier}, чтобы убедиться, что все потоки завершили сортировку своих векторов :
\par  Объединяем отсортированные векторы с помощью функции слияния \cpp{merge}


\subsection{Реализация TBB}

\par Используем \cpp{tbb::parallel\_for} для параллельного выполнения цикла. Каждый поток сортирует свой вектор с помощью последовательной сортировки Шелла (\cpp{shell\_sort}).
\par В TBB не требуется явная синхронизация барьером, так как \cpp{tbb::parallel\_for} обеспечивает автоматическую синхронизацию.
\par Объединяем отсортированные векторы с помощью функции слияния \cpp{merge}.


\section{Исследование и анализ}

\subsection{Подтверждение корректности}

\par Для подтверждения корректности работы программы были написаны тесты с использованием Google Test: созданы по 5 тестов проверки функциональности для каждой реализации. Также тесты на замер времени работы всего пайплайна и работы задачи.

\subsection{Результаты экспериментов}

\par Для проведения экспериментов по вычислению эффективности работы разных реализаций программы использовалась система со следующей конфигурацией:

\begin{itemize}
    \item Процессор: AMD Ryzen 5 5625U with Radeon Graphics 2.30 GHz 6 ядер 12 потоков
    \item Оперативная память: 16.00 ГБ
    \item ОС: Windows 11 Домашняя
\end{itemize}

\par В качестве тестовых данных использовались рандомные массивы различных размеров, сгенерированные специальной функцией:
\begin{lstlisting}[language=C++]
std::vector<int> ShellTaskSequential::give_random_vector(int size, int min, int max);
\end{lstlisting}

\par Для проверки корректности работы сортировки использовалась функция:

\begin{lstlisting}[language=C++]
bool ShellTaskSequential::CheckSort(std::vector<int> input);
\end{lstlisting}
 -- проверяет упорядоченность элементов в массиве.
\par Для функциональных тестов использовались случайные массивы размерами от 10 до 100. Функциональные тесты для всех реализаций успешно прошли проверку.

\par Для тестов производительности пайплайна использовались массивы размером $2 \cdot 10^6$. Для тестов производительности задачи использовались массивы размером $8 \cdot 10^6$. В результате работы тестов были получены следующие результаты:

\begin{table}[h!]
    \begin{center}
        \begin{tabular}{ |c|c|c|c| } 
        \hline
        Версия программы & Общ. время, с. & Сравнение с послед. & Эффективность  \\
        \hline
        Последовательная & 5,5249009 & 1 & 0,25 \\
        OpenMP & 2,4433778 & 2,4151213 & 0,293392243 \\
        TBB & 2,56223 & 2,187295179 & 0,270326897 \\
        \hline
        \end{tabular}
    \end{center}
    \caption{Тесты pipeline}
\end{table}

\begin{table}[h]
    \begin{center}
        \begin{tabular}{ |c|c|c|c| } 
        \hline
        Версия программы & Общ. время, с. & Сравнение с послед. & Эффективность  \\
        \hline
        Последовательная & 1,7987948 & 1 & 0,25 \\
        OpenMP & 1,3814713 & 1,3379727 & 0,155746617 \\
        TBB & 1,1305301 & 1,602312177 & 0,198038286 \\
        \hline
        \end{tabular}
    \end{center}
    \caption{Тесты task run}
\end{table}

\subsection{Выводы}

\par Работу всей программы удалось ускорить более чем в 2 раза;
\par Версия с OpenMP работает немного быстрее, чем версия с TBB;
\par Однако задача работает быстрее на TBB.

\par Отсюда можно сделать вывод, что параллелизация алгоритма сортировки Шелла с простым слиянием работает эффективно.

\newpage

\section{Заключение}

\par В результате данной работы удалось изучить алгоритмы сортировки Шелла и простого слияния. Были реализованы как последовательный вариант сортировки, так и параллельный с помощью технологий OpenMP и TBB. Параллельная реализация продемонстрировала свою эффективность. 

\newpage

\section{Литература}
\begin{enumerate}
\item Сортировка Шелла :: Shell sort - Электронный ресурс. URL: \newline \url{http://algolab.valemak.com/shell}
\item Habr - Электронный ресурс. URL: \newline \url{https://habr.com/ru/post/479202/}
\item Educative - Электронный ресурс. URL: \newline \url{https://www.educative.io/blog/modern-multithreading-and-concurrency-in-cpp}
\item А.В. Сысоев, И.Б. Мееров, А.А. Сиднев «Средства разработки параллельных программ для систем с общей памятью. Библиотека Intel Threading Building Blocks». Нижний Новгород, 2007, 128 с. 
\item А.В. Сысоев, И.Б. Мееров, А.Н. Свистунов, А.Л. Курылев, А.В. Сенин, А.В. Шишков, К.В. Корняков, А.А. Сиднев «Параллельное программирование в системах с общей
памятью. Инструментальная поддержка». Нижний Новгород, 2007, 110 с. 
\end{enumerate}

\newpage

\section{Приложение}

\begin{lstlisting}[language=C++]
std::vector<int> ShellTaskSequential::ShellSort(const std::vector<int>& input) {
  std::vector<int> vec(input);

  for (int interval = static_cast<int>(vec.size()) / 2; interval > 0; interval /= 2) {
    for (int i = interval; i < static_cast<int>(vec.size()); i++) {
      int tmp = vec[i];
      int j = i;
      for (; j >= interval && vec[j - interval] > tmp; j -= interval) {
        vec[j] = vec[j - interval];
      }
      vec[j] = tmp;
    }
  }

  return vec;
}

bool ShellTaskSequential::CheckSort(std::vector<int> input) { return std::is_sorted(input.begin(), input.end()); }

std::vector<int> ShellSequential::generate_random_vector(int size, int min, int max) {
  std::random_device rnd_device;
  std::mt19937 mersenne_engine{rnd_device()};
  std::uniform_int_distribution<int> dist{min, max};

  auto gen = [&dist, &mersenne_engine]() { return dist(mersenne_engine); };

  std::vector<int> vec(size);
  generate(begin(vec), end(vec), gen);

  return vec;
}

std::vector<int> ShellTaskOMP::merge(const std::vector<std::vector<int>>& chunks) {
  std::vector<int> res;

  // Merge the sorted chunks
  for (const auto& chunk : chunks) {
    // Merge the current chunk with the result vector
    std::vector<int> merged;
    merged.reserve(res.size() + chunk.size());
    auto resIter = res.begin();
    auto chunkIter = chunk.begin();

    while (resIter != res.end() && chunkIter != chunk.end()) {
      if (*resIter < *chunkIter) {
        merged.push_back(*resIter);
        ++resIter;
      } else {
        merged.push_back(*chunkIter);
        ++chunkIter;
      }
    }

    // Copy remaining elements from result vector
    while (resIter != res.end()) {
      merged.push_back(*resIter);
      ++resIter;
    }

    // Copy remaining elements from current chunk
    while (chunkIter != chunk.end()) {
      merged.push_back(*chunkIter);
      ++chunkIter;
    }

    // Update result vector with merged vector
    res = std::move(merged);
  }

  return res;
}

void ShellTaskOMP::ShellSort_Parallel(std::vector<int>& input) {
  int numProcs = omp_get_num_procs();
  int size = static_cast<int>(input.size());
  int chunkSize = size / numProcs;
  int remainder = size % numProcs;

  std::vector<std::vector<int>> chunks(numProcs);

#pragma omp parallel for
  for (int i = 0; i < numProcs; ++i) {
    int startIdx = i * chunkSize;
    int endIdx = startIdx + chunkSize;
    if (i == numProcs - 1) {
      endIdx += remainder;  // Add remaining elements to the last chunk
    }
    chunks[i].assign(input.begin() + startIdx, input.begin() + endIdx);
    shell_sort(chunks[i]);
  }

#pragma omp barrier

  input = merge(chunks);
}

void ShellTaskTBB::ShellSort_Parallel(std::vector<int>& input) {
  int numProcs = 4;
  int size = static_cast<int>(input.size());
  int chunkSize = size / numProcs;
  int remainder = size % numProcs;

  std::vector<std::vector<int>> chunks(numProcs);

  tbb::parallel_for(0, numProcs, [&](int i) {
    int startIdx = i * chunkSize;
    int endIdx = startIdx + chunkSize;
    if (i == numProcs - 1) {
      endIdx += remainder;  // Add remaining elements to the last chunk
    }
    chunks[i].assign(input.begin() + startIdx, input.begin() + endIdx);
    shell_sort(chunks[i]);
  });

  // Barrier is not needed in TBB

  input = merge(chunks);
}
\end{lstlisting}

\end{document}